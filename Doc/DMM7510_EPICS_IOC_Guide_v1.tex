\documentclass[openany]{article}
\usepackage[a4paper,margin=1in,bottom=1.5in]{geometry} % define margins. Bottom margin is used to lift a little bit the page number.
\usepackage[english]{babel} % document language is english
\usepackage{tikz} % for drawing (currently not used).
\usepackage{graphicx} % for including images
\usepackage[export]{adjustbox}
\usepackage{fancyhdr} % used for creating headers and footers. only used in title page in this document.
\usepackage{tabularx} % creation of more complex tables
\usepackage{longtable} % tables can span multiple pages
\usepackage{array} % allow elements of tabular environment to have vertical alignment, e.g., center alignment.
\usepackage{nameref} % make it possible to reference by name
\usepackage{hyperref} % allow hiperlinks (links to other document parts and extern links)
\usepackage{etoc} % used for generation of section local table of contents
\usepackage{placeins}
\usepackage{siunitx} % SI units package
\usepackage{enumitem} % allows removing space between list items

% Define graphics path
\graphicspath{{figs/}}

% Configure the cross reference hyper links color
\hypersetup{
    colorlinks=true,
    linkcolor=blue,
}

\newcolumntype{C}{>{\centering\arraybackslash}X} % new column type for tabularx
						 % centered (\centering), adjust width in order to fill table width (X type)

% Configure header in 'titlepage'
%\pagestyle{fancy}
%\lhead{\includegraphics[width=4.5cm]{logo_cnpem}}
%\rhead{\includegraphics[width=4cm]{logo_lnls}}
%\renewcommand{\headrulewidth}{0pt}
%\setlength{\headheight}{52pt}
% Clean footer
%\fancyfoot{}

% increase table height factor a little bit (taller cells)
%\renewcommand{\arraystretch}{1.5}

%==== Begin DOCUMENT ====
\begin{document}

%--- Begin title page ---
\begin{titlepage}

% Add header to this page
%\thispagestyle{fancy}

% Center elements
\begin{center}

% title of title page
\topskip0pt % perfectly centered
\vspace*{\fill}
\textbf{\Huge DMM7510 EPICS IOC User Guide}\\[20pt]
\textbf{\Huge Version 1.0}\\[20pt]
\textbf{\Huge June/2017}
\vspace*{\fill}

% footer of title page
\vfill
\textbf{Beam Diagnostics Group (DIG)}\\[5pt]
\textbf{Brazilian Synchrotron Light Laboratory (LNLS)}\\[5pt]
\textbf{Brazilian Center for Research in Energy and Materials (CNPEM)}
\end{center}

\end{titlepage}
%--- End of title page ---

\newpage
\pagestyle{plain} % restore default page style

%--- Table of contents ---
\tableofcontents
\listoftables

\newpage
%--- Section: Introduction ---
\section{Introduction}

	\paragraph{} This IOC provides most of the DMM7510 parameters as low level PVs. Its purpose is to serve as a base for bulding higher level applications that make use of the DMM7510 digital multimeter.

%--- Section: PV List ---
\section{PV List}

	\newcolumntype{A}{p{3cm}} % new column type
	\newcolumntype{B}{p{5cm}} % new column type
	\newcolumntype{P}{p{7cm}} % new column type
	\renewcommand{\arraystretch}{2}

	% TABLE: Measure/Digitize Function
	\begin{longtable}{A B P}
		\caption{Measure/Digitize Function} \\ \hline
		\bfseries Name & \bfseries Data Type & \bfseries Description \\ \hline
		%---
		MeasFnc-Sel & enum \{\begin{itemize}[noitemsep]
					\small
					\item[] DC\_VOLTAGE
					\item[] AC\_VOLTAGE
					\item[] DC\_CURRENT
					\item[] AC\_CURRENT
					\item[] RESISTANCE
					\item[] 4W\_RESISTANCE
					\item[] DIODE
					\item[] CAPACITANCE
					\item[] TEMPERATURE
					\item[] CONTINUITY
					\item[] ACV\_FREQUENCY
					\item[] ACV\_PERIOD
					\item[] DCV\_RATIO
					\end{itemize}\} &
				\begin{tabular}{P}
					TSP command: dmm.measure.func \\
					This attribute selects the active measure function.
				\end{tabular} \\

		MeasFnc-Sts & enum \{\begin{itemize}[noitemsep]
					\small
					\item[] NONE
					\item[] DC\_VOLTAGE
					\item[] AC\_VOLTAGE
					\item[] DC\_CURRENT
					\item[] AC\_CURRENT
					\item[] RESISTANCE
					\item[] 4W\_RESISTANCE
					\item[] DIODE
					\item[] CAPACITANCE
					\item[] TEMPERATURE
					\item[] CONTINUITY
					\item[] ACV\_FREQUENCY
					\item[] ACV\_PERIOD
					\item[] DCV\_RATIO
					\end{itemize}\} & 
				\begin{tabular}{P}
					TSP command: dmm.measure.func \\
					This attribute shows the active measure function.
				\end{tabular} \\ \hline
		%---
		DigtzeFnc-Sel & enum \{\begin{itemize}[noitemsep]
					\small
					\item[] DIGITIZE\_VOLTAGE
					\item[] DIGITIZE\_CURRENT
					\end{itemize}\} &
				\begin{tabular}{P}
					TSP command: dmm.digitize.func \\
					This attribute determines which digitize function is active.
				\end{tabular} \\

		DigtzeFnc-Sts & enum \{\begin{itemize}[noitemsep]
					\small
					\item[] NONE
					\item[] DIGITIZE\_VOLTAGE
					\item[] DIGITIZE\_CURRENT
					\end{itemize}\} & 
				\begin{tabular}{P}
					TSP command: dmm.digitize.func \\
					This attribute shows which digitize function is active.
				\end{tabular} \\ \hline
		Function-Sts & enum \{\begin{itemize}[noitemsep]
					\item[] DC\_VOLTAGE
					\item[] AC\_VOLTAGE
					\item[] DC\_CURRENT
					\item[] AC\_CURRENT
					\item[] RESISTANCE
					\item[] 4W\_RESISTANCE
					\item[] DIODE
					\item[] CAPACITANCE
					\item[] TEMPERATURE
					\item[] CONTINUITY
					\item[] ACV\_FREQUENCY
					\item[] ACV\_PERIOD
					\item[] DCV\_RATIO
					\item[] DIGITIZE\_VOLTAGE
					\item[] DIGITIZE\_CURRENT
					\end{itemize}\} &
				\begin{tabular}{P}
					TSP command: No command \\
					This attribute shows which measurement or digitize function is active.
				\end{tabular} \\ \hline
		%---
	\end{longtable}
	\begin{longtable}{A B P}
		\caption{Measurement settings} \\ \hline
		\bfseries Name & \bfseries Data Type & \bfseries Description \\ \hline
		%---
		MeasApert-SP & \begin{tabular}{B}
					float \\
					unit: second
				\end{tabular} & 
				\begin{tabular}{P}
					TSP command: dmm.measure.aperture \\
					This attribute determines the aperture setting for the selected measurement function.
				\end{tabular} \\

		MeasApert-RB & \begin{tabular}{B}
					float \\
					unit: second
				\end{tabular} & 
				\begin{tabular}{P}
					TSP command: dmm.measure.aperture \\
					This attribute shows the aperture setting for the selected measurement function.
				\end{tabular} \\ \hline
		%---
		MeasNPLC-SP & \begin{tabular}{B}
					float
				\end{tabular} & 
				\begin{tabular}{P}
					TSP command: dmm.measure.nplc \\
					This attribute sets the time that the input signal is measured for the selected function.
				\end{tabular} \\

		MeasNPLC-RB & \begin{tabular}{B}
					float
				\end{tabular} & 
				\begin{tabular}{P}
					TSP command: dmm.measure.nplc \\
					This attribute shows the time that the input signal is measured for the selected function.
				\end{tabular} \\ \hline
		%---
		MeasCount-SP & \begin{tabular}{B}
					long \\
					Min=1 \\
					Max=1000000
				\end{tabular} & 
				\begin{tabular}{P}
					TSP command: dmm.measure.count \\
					This attribute sets the number of measurements to make when a measurement is requested.
				\end{tabular} \\

		MeasCount-RB & \begin{tabular}{B}
					long
				\end{tabular} & 
				\begin{tabular}{P}
					TSP command: dmm.measure.count \\
					This attribute shows the number of measurements to make when a measurement is requested.
				\end{tabular} \\ \hline
		%---
		MRange-SP & \begin{tabular}{B}
					float
				\end{tabular} & 
				\begin{tabular}{P}
					TSP command: dmm.measure.range \\
					This attribute determines the positive full-scale measure range.
				\end{tabular} \\

		MRange-RB & \begin{tabular}{B}
					float
				\end{tabular} & 
				\begin{tabular}{P}
					TSP command: dmm.measure.range \\
					This attribute shows the positive full-scale measure range.
				\end{tabular} \\ \hline
		%---
		MAutoRange-Sel & bool\{\begin{itemize}[noitemsep]
					\small
					\item[] OFF
					\item[] ON
				\end{itemize}\} & 
				\begin{tabular}{P}
					TSP command: dmm.measure.autorange \\
					This attribute determines if the measurement range is set manually or automatically for the selected function.
				\end{tabular} \\

		MAutoRange-Sts & bool\{\begin{itemize}[noitemsep]
					\small
					\item[] OFF
					\item[] ON
				\end{itemize}\} & 
				\begin{tabular}{P}
					TSP command: dmm.measure.autorange \\
					This attribute shows if the measurement range is set manually or automatically for the selected function.
				\end{tabular} \\ \hline
		%---
		AutoZero-Sel & bool\{\begin{itemize}[noitemsep]
					\small
					\item[] OFF
					\item[] ON
				\end{itemize}\} & 
				\begin{tabular}{P}
					TSP command: dmm.measure.autozero.enable \\
					This attribute enables or disables automatic updates to the internal reference measurements (autozero) of the instrument.
				\end{tabular} \\

		AutoZero-Sts & bool\{\begin{itemize}[noitemsep]
					\small
					\item[] OFF
					\item[] ON
				\end{itemize}\} & 
				\begin{tabular}{P}
					TSP command: dmm.measure.autozero.enable \\
					This attribute shows if automatic updates to the internal reference measurements (autozero) of the instrument are enabled.
				\end{tabular} \\ \hline
		%---
		AZeroOnce-Cmd & bool\{\begin{itemize}[noitemsep]
					\small
					\item[] OFF
					\item[] ON
				\end{itemize}\} & 
				\begin{tabular}{P}
					TSP command: dmm.measure.autozero.once() \\
					Sending 1 or \emph{ON} causes the instrument to refresh the reference and zero measurements once.
				\end{tabular} \\
		%---
		MeasAutoDly-Sel & bool\{\begin{itemize}[noitemsep]
					\small
					\item[] OFF
					\item[] ON
				\end{itemize}\} & 
				\begin{tabular}{P}
					TSP command: dmm.measure.autodelay \\
					This attribute enables or disables the automatic delay that occurs before each measurement.
				\end{tabular} \\

		MeasAutoDly-Sts & bool\{\begin{itemize}[noitemsep]
					\small
					\item[] OFF
					\item[] ON
				\end{itemize}\} & 
				\begin{tabular}{P}
					TSP command: dmm.measure.autodelay \\
					This attribute shows if the automatic delay that occurs before each measurement is enabled.
				\end{tabular} \\ \hline
		%---
		MeasImpedance-Sel & bool\{\begin{itemize}[noitemsep]
					\small
					\item[] AUTO
					\item[] 10MOhm
				\end{itemize}\} & 
				\begin{tabular}{P}
					TSP command: dmm.measure.inputimpedance \\
					This attribute determines when the 10 MΩ input divider is enabled for the seslected measure function.
				\end{tabular} \\

		MeasImpedance-Sts & bool\{\begin{itemize}[noitemsep]
					\small
					\item[] AUTO
					\item[] 10MOhm
				\end{itemize}\} & 
				\begin{tabular}{P}
					TSP command: dmm.measure.inputimpedance \\
					This attribute shows when the 10 MΩ input divider is enabled for the selected measure function.
				\end{tabular} \\ \hline
		%---
		MeasLineSync-Sel & bool\{\begin{itemize}[noitemsep]
					\small
					\item[] OFF
					\item[] ON
				\end{itemize}\} & 
				\begin{tabular}{P}
					TSP command: dmm.measure.linesync \\
					This attribute determines if line synchronization is used during the measurement.
				\end{tabular} \\

		MeasLineSync-Sts & bool\{\begin{itemize}[noitemsep]
					\small
					\item[] OFF
					\item[] ON
				\end{itemize}\} & 
				\begin{tabular}{P}
					TSP command: dmm.measure.linesync \\
					This attribute shows if line synchronization is used during the measurement.
				\end{tabular} \\ \hline
		%---
		MeasStim-Sel & enum\{\begin{itemize}[noitemsep]
					\small
					\item[] EVENT\_NONE
					\item[] EVENT\_DISPLAY
					\item[] EVENT\_NOTIFY\textless n\textgreater
					\item[] ($1\leq n\leq 8$)
					\item[] EVENT\_COMMAND
					\item[] EVENT\_DIGIO\textless n\textgreater
					\item[] ($1\leq n\leq 6$)
					\item[] EVENT\_TSPLINK\textless n\textgreater
					\item[] ($1\leq n\leq 3$)
					\item[] EVENT\_LAN\textless n\textgreater
					\item[] ($1\leq n\leq 8$)
					\item[] EVENT\_BLENDER\textless n\textgreater 
					\item[] ($1\leq n\leq 2$)
					\item[] EVENT\_TIMER\textless n\textgreater
					\item[] ($1\leq n\leq 4$)
					\item[] EVENT\_ANALOGTRIGGER
					\item[] EVENT\_EXTERNAL
				\end{itemize}\} & 
				\begin{tabular}{P}
					TSP command: dmm.trigger.measure.stimulus \\
					This attribute sets the instrument to make a measurement when it detects the specified trigger event.
				\end{tabular} \\

		MeasStim-Sts & enum\{\begin{itemize}[noitemsep]
					\small
					\item[] EVENT\_NONE
					\item[] EVENT\_DISPLAY
					\item[] EVENT\_NOTIFY\textless n\textgreater
					\item[] ($1\leq n\leq 8$)
					\item[] EVENT\_COMMAND
					\item[] EVENT\_DIGIO\textless n\textgreater
					\item[] ($1\leq n\leq 6$)
					\item[] EVENT\_TSPLINK\textless n\textgreater
					\item[] ($1\leq n\leq 3$)
					\item[] EVENT\_LAN\textless n\textgreater
					\item[] ($1\leq n\leq 8$)
					\item[] EVENT\_BLENDER\textless n\textgreater
					\item[] ($1\leq n\leq 2$)
					\item[] EVENT\_TIMER\textless n\textgreater
					\item[] ($1\leq n\leq 4$)
					\item[] EVENT\_ANALOGTRIGGER
					\item[] EVENT\_EXTERNAL
				\end{itemize}\} & 
				\begin{tabular}{P}
					TSP command: dmm.trigger.measure.stimulus \\
					This attribute shows the instrument configured measurement trigger event for the selected measure function.
				\end{tabular} \\ \hline
		%---
		MATrMode-Sel & enum\{\begin{itemize}[noitemsep]
					\small
					\item[] OFF
					\item[] Edge
					\item[] Pulse
					\item[] Window
				\end{itemize}\} & 
				\begin{tabular}{P}
					TSP command: dmm.digitize.analogtrigger.mode \\
					This attribute configures the type of signal behavior that can generate an analog trigger event.
				\end{tabular} \\

		MATrMode-Sts & enum\{\begin{itemize}[noitemsep]
					\small
					\item[] OFF
					\item[] Edge
					\item[] Pulse
					\item[] Window
				\end{itemize}\} & 
				\begin{tabular}{P}
					TSP command: dmm.digitize.analogtrigger.mode \\
					This attribute shows the configured type of signal behavior that can generate an analog trigger event.
				\end{tabular} \\ \hline
		%---
		MATrEdgeSlp-Sel & bool\{\begin{itemize}[noitemsep]
					\small
					\item[] Rising
					\item[] Falling
				\end{itemize}\} & 
				\begin{tabular}{P}
					TSP command: dmm.digitize.analogtrigger.edge.slope \\
					This attribute defines the slope of the analog trigger edge.
				\end{tabular} \\

		MATrEdgeSlp-Sts & bool\{\begin{itemize}[noitemsep]
					\small
					\item[] Rising
					\item[] Falling
				\end{itemize}\} & 
				\begin{tabular}{P}
					TSP command: dmm.digitize.analogtrigger.edge.slope \\
					This attribute shows the slope of the analog trigger edge.
				\end{tabular} \\ \hline
		%---
		MATrEdgeLvl-SP & \begin{tabular}{B}
					float 
				\end{tabular} & 
				\begin{tabular}{P}
					TSP command: dmm.digitize.analogtrigger.edge.level \\
					This attribute defines the signal level that generates the analog trigger event for the edge trigger mode.
				\end{tabular} \\

		MATrEdgeLvl-RB & \begin{tabular}{B}
					float 
				\end{tabular} & 
				\begin{tabular}{P}
					TSP command: dmm.digitize.analogtrigger.edge.level \\
					This attribute shows the signal level that generates the analog trigger event for the edge trigger mode.
				\end{tabular} \\ \hline
		%---
		MATrHFR-Sel & bool\{\begin{itemize}[noitemsep]
					\small
					\item[] OFF
					\item[] ON
				\end{itemize}\} & 
				\begin{tabular}{P}
					TSP command: dmm.measure.analogtrigger.highfreqreject \\
					This attribute enables or disables high frequency rejection on analog trigger events.
				\end{tabular} \\

		MATrHFR-Sts & bool\{\begin{itemize}[noitemsep]
					\small
					\item[] OFF
					\item[] ON
				\end{itemize}\} & 
				\begin{tabular}{P}
					TSP command: dmm.measure.analogtrigger.highfreqreject \\
					This attribute shows if high frequency rejection on analog trigger events is enabled.
				\end{tabular} \\ \hline
		%---
		MATrPulCond-Sel & bool\{\begin{itemize}[noitemsep]
					\small
					\item[] Greater
					\item[] Less
				\end{itemize}\} & 
				\begin{tabular}{P}
					TSP command: dmm.measure.analogtrigger.pulse.condition \\
					This attribute defines if the pulse must be greater than or less than the pulse width before an analog trigger is generated.
				\end{tabular} \\

		MATrPulCond-Sts & bool\{\begin{itemize}[noitemsep]
					\small
					\item[] Greater
					\item[] Less
				\end{itemize}\} & 
				\begin{tabular}{P}
					TSP command: dmm.measure.analogtrigger.pulse.condition \\
					This attribute shows if the pulse must be greater than or less than the pulse width before an analog trigger is generated.
				\end{tabular} \\ \hline
		%---
		MATrPulPol-Sel & bool\{\begin{itemize}[noitemsep]
					\small
					\item[] Above
					\item[] Below
				\end{itemize}\} & 
				\begin{tabular}{P}
					TSP command: dmm.measure.analogtrigger.pulse.polarity \\
					This attribute defines the polarity of the pulse that generates an analog trigger event.
				\end{tabular} \\

		MATrPulPol-Sts & bool\{\begin{itemize}[noitemsep]
					\small
					\item[] Above
					\item[] Below
				\end{itemize}\} & 
				\begin{tabular}{P}
					TSP command: dmm.measure.analogtrigger.pulse.polarity \\
					This attribute shows the polarity of the pulse that generates an analog trigger event.
				\end{tabular} \\ \hline
		%---
		MATrPulLvl-SP & \begin{tabular}{B}
					float
				\end{tabular} & 
				\begin{tabular}{P}
					TSP command: dmm.measure.analogtrigger.pulse.level \\
					This attribute defines the pulse level that generates an analog trigger event.
				\end{tabular} \\

		MATrPulLvl-RB & \begin{tabular}{B}
					float
				\end{tabular} & 
				\begin{tabular}{P}
					TSP command: dmm.measure.analogtrigger.pulse.level \\
					This attribute shows the pulse level that generates an analog trigger event.
				\end{tabular} \\ \hline
		%---
		MATrPulWidth-SP & \begin{tabular}{B}
					float \\
					Min=0.000001 \\
					Max=0.04
				\end{tabular} & 
				\begin{tabular}{P}
					TSP command: dmm.measure.analogtrigger.pulse.width \\
					This attribute defines the threshold value for the pulse width.
				\end{tabular} \\

		MATrPulWidth-RB & \begin{tabular}{B}
					float
				\end{tabular} & 
				\begin{tabular}{P}
					TSP command: dmm.measure.analogtrigger.pulse.width \\
					This attribute defines the threshold value for the pulse width.
				\end{tabular} \\ \hline
		%---
		MATrWindHigh-SP & \begin{tabular}{B}
					float
				\end{tabular} & 
				\begin{tabular}{P}
					TSP command: dmm.measure.analogtrigger.window.levelhigh \\
					This attribute defines the upper boundary of the analog trigger window.
				\end{tabular} \\

		MATrWindHigh-RB & \begin{tabular}{B}
					float
				\end{tabular} & 
				\begin{tabular}{P}
					TSP command: dmm.measure.analogtrigger.window.levelhigh \\
					This attribute shows the upper boundary of the analog trigger window.
				\end{tabular} \\ \hline
		%---
		MATrWindLow-SP & \begin{tabular}{B}
					float
				\end{tabular} & 
				\begin{tabular}{P}
					TSP command: dmm.measure.analogtrigger.window.levellow \\
					This attribute defines the lower boundary of the analog trigger window.
				\end{tabular} \\

		MATrWindLow-RB & \begin{tabular}{B}
					float
				\end{tabular} & 
				\begin{tabular}{P}
					TSP command: dmm.measure.analogtrigger.window.levellow \\
					This attribute shows the lower boundary of the analog trigger window.
				\end{tabular} \\ \hline
		%---
		MATrWindDir-Sel & bool\{\begin{itemize}[noitemsep]
					\small
					\item[] Enter
					\item[] Leave
				\end{itemize}\} & 
				\begin{tabular}{P}
					TSP command: dmm.measure.analogtrigger.window.direction \\
					This attribute defines if the analog trigger occurs when the signal enters or leaves the defined upper and lower analog signal level boundaries.
				\end{tabular} \\

		MATrWindDir-Sts & bool\{\begin{itemize}[noitemsep]
					\small
					\item[] Enter
					\item[] Leave
				\end{itemize}\} & 
				\begin{tabular}{P}
					TSP command: dmm.measure.analogtrigger.window.direction \\
					This attribute shows if the analog trigger occurs when the signal enters or leaves the defined upper and lower analog signal level boundaries.
				\end{tabular} \\ \hline
		%---
		MRelOffEnbl-Sel & bool\{\begin{itemize}[noitemsep]
					\small
					\item[] OFF
					\item[] ON
				\end{itemize}\} & 
				\begin{tabular}{P}
					TSP command: dmm.measure.rel.enable \\
					This attribute enables or disables the application of a relative offset value to the measurement for the selected measure function.
				\end{tabular} \\

		MRelOffEnbl-Sts & bool\{\begin{itemize}[noitemsep]
					\small
					\item[] OFF
					\item[] ON
				\end{itemize}\} & 
				\begin{tabular}{P}
					TSP command: dmm.measure.rel.enable \\
					This attribute enables or disables the application of a relative offset value to the measurement for the selected measure function.
				\end{tabular} \\ \hline
		%---
		MRelOffAcq-Cmd & bool\{\begin{itemize}[noitemsep]
					\small
					\item[] OFF
					\item[] ON
				\end{itemize}\} & 
				\begin{tabular}{P}
					TSP command: dmm.measure.rel.acquire() \\
					When set to 1 or \emph{ON}, this function acquires a measurement and stores it as the relative offset value.
				\end{tabular} \\
		%---
		MRelOff-SP & \begin{tabular}{B}
					float
				\end{tabular} & 
				\begin{tabular}{P}
					TSP command: dmm.measure.rel.level \\
					This attribute sets the relative offset value for the selected measure function.
				\end{tabular} \\

		MRelOff-RB & \begin{tabular}{B}
					float
				\end{tabular} & 
				\begin{tabular}{P}
					TSP command: dmm.measure.rel.level \\
					This attribute shows the relative offset value for the selected measure function.
				\end{tabular} \\ \hline
		%---
		MMathEnbl-Sel & bool\{\begin{itemize}[noitemsep]
					\small
					\item[] OFF
					\item[] ON
				\end{itemize}\} & 
				\begin{tabular}{P}
					TSP command: dmm.measure.math.enable \\
					This attribute enables or disables math operations on measurements for the selected measurement function.
				\end{tabular} \\

		MMathEnbl-Sts & bool\{\begin{itemize}[noitemsep]
					\small
					\item[] OFF
					\item[] ON
				\end{itemize}\} & 
				\begin{tabular}{P}
					TSP command: dmm.measure.math.enable \\
					This attribute shows if math operations on measurements for the selected measurement function are enabled.
				\end{tabular} \\ \hline
		%---
		MMathOp-Sel & enum\{\begin{itemize}[noitemsep]
					\small
					\item[] y=mx+b
					\item[] Percent
					\item[] Reciprocal
				\end{itemize}\} & 
				\begin{tabular}{P}
					TSP command: dmm.measure.math.format \\
					This attribute specifies which math operation is performed on measurements when math operations are enabled.
				\end{tabular} \\

		MMathOp-Sts & enum\{\begin{itemize}[noitemsep]
					\small
					\item[] y=mx+b
					\item[] Percent
					\item[] Reciprocal
				\end{itemize}\} & 
				\begin{tabular}{P}
					TSP command: dmm.measure.math.format \\
					This attribute shows which math operation is performed on measurements when math operations are enabled.
				\end{tabular} \\ \hline
		%---
		MMathBFactor-SP & \begin{tabular}{B}
					float \\
					Min=-1000000000000 \\
					Max=1000000000000
				\end{tabular} & 
				\begin{tabular}{P}
					TSP command: dmm.measure.math.mxb.bfactor \\
					This attribute specifies the offset, b, for the y = mx + b operation.
				\end{tabular} \\

		MMathBFactor-RB & \begin{tabular}{B}
					float
				\end{tabular} & 
				\begin{tabular}{P}
					TSP command: dmm.measure.math.mxb.bfactor \\
					This attribute shows the offset, b, for the y = mx + b operation.
				\end{tabular} \\ \hline
		%---
		MMathMFactor-SP & \begin{tabular}{B}
					float \\
					Min=-1000000000000 \\
					Max=1000000000000
				\end{tabular} & 
				\begin{tabular}{P}
					TSP command: dmm.measure.math.mxb.mfactor \\
					This attribute specifies the scale factor, m, for the y = mx + b math operation.
				\end{tabular} \\

		MMathMFactor-RB & \begin{tabular}{B}
					float
				\end{tabular} & 
				\begin{tabular}{P}
					TSP command: dmm.measure.math.mxb.mfactor \\
					This attribute shows the scale factor, m, for the y = mx + b math operation.
				\end{tabular} \\ \hline
		%---
		MMathPercRef-SP & \begin{tabular}{B}
					float \\
					Min=-1000000000000 \\
					Max=1000000000000
				\end{tabular} & 
				\begin{tabular}{P}
					TSP command: dmm.measure.math.percent \\
					This attribute specifies the reference constant that is used when math operations are set to percent.
				\end{tabular} \\

		MMathPercRef-RB & \begin{tabular}{B}
					float
				\end{tabular} & 
				\begin{tabular}{P}
					TSP command: dmm.measure.math.percent \\
					This attribute shows the reference constant that is used when math operations are set to percent.
				\end{tabular} \\ \hline
		%---
		FilterEnbl-Sel & bool\{\begin{itemize}[noitemsep]
					\item[] OFF
					\item[] ON
				\end{itemize}\} & 
				\begin{tabular}{P}
					TSP command: dmm.measure.filter.enable \\
					This attribute enables or disables the averaging filter for measurements of the selected function.
				\end{tabular} \\

		FilterEnbl-Sts & bool\{\begin{itemize}[noitemsep]
					\item[] OFF
					\item[] ON
				\end{itemize}\} & 
				\begin{tabular}{P}
					TSP command: dmm.measure.filter.enable \\
					This attribute shows if the averaging filter for measurements of the selected function is enabled.
				\end{tabular} \\ \hline
		%---
		FilterCount-SP & \begin{tabular}{B}
					long \\
					Min=1 \\
					Max=100
				\end{tabular} & 
				\begin{tabular}{P}
					TSP command: dmm.measure.filter.count \\
					This attribute sets the number of measurements that are averaged when filtering is enabled.
				\end{tabular} \\

		FilterCount-RB & \begin{tabular}{B}
					long
				\end{tabular} & 
				\begin{tabular}{P}
					TSP command: dmm.measure.filter.count \\
					This attribute shows the number of measurements that are averaged when filtering is enabled.
				\end{tabular} \\ \hline
		%---
		FilterTyp-Sel & bool\{\begin{itemize}[noitemsep]
					\item[] Repeat
					\item[] Moving
				\end{itemize}\} & 
				\begin{tabular}{P}
					TSP command: dmm.measure.filter.type \\
					This attribute defines the type of averaging filter that is used for the selected function when the filter is enabled.
				\end{tabular} \\

		FilterTyp-Sts & bool\{\begin{itemize}[noitemsep]
					\item[] Repeat
					\item[] Moving
				\end{itemize}\} & 
				\begin{tabular}{P}
					TSP command: dmm.measure.filter.type \\
					This attribute shows the type of averaging filter that is used for the selected function when the filter is enabled.
				\end{tabular} \\ \hline
		%---
		FilterWind-SP & \begin{tabular}{B}
					long \\
					Min=0 \\
					Max=10
				\end{tabular} & 
				\begin{tabular}{P}
					TSP command: dmm.measure.filter.window \\
					This attribute sets the window for the averaging filter that is used for measurements for the selected function.
				\end{tabular} \\

		FilterWind-RB & \begin{tabular}{B}
					long
				\end{tabular} & 
				\begin{tabular}{P}
					TSP command: dmm.measure.filter.window \\
					This attribute shows the window for the averaging filter that is used for measurements for the selected function.
				\end{tabular} \\ \hline
	\end{longtable}
	
	% TABLE: Digitize  Settings
	\begin{longtable}{A B P}
		\caption{Digitize settings} \\ \hline
		\bfseries Name & \bfseries Data Type & \bfseries Description \\ \hline
		%---
		DigtzeApert-SP & \begin{tabular}{B}
					long \\
					Min=1 \\
					Max=1000 \\
					unit: \SI{}{\micro\second}
				\end{tabular} & 
				\begin{tabular}{P}
					TSP command: dmm.digitize.aperture \\
					This attribute determines the aperture setting for the selected function.
				\end{tabular} \\

		DigtzeApert-RB & \begin{tabular}{B}
					long
					unit: \SI{}{\micro\second}
				\end{tabular} & 
				\begin{tabular}{P}
					TSP command: dmm.digitize.aperture \\
					This attribute shows the aperture setting for the selected function.
				\end{tabular} \\ \hline
		%---
		DigtzeApertAuto-Cmd & bool\{\begin{itemize}[noitemsep]
					\small
					\item[] OFF
					\item[] ON
				\end{itemize}\} & 
				\begin{tabular}{P}
					TSP command: dmm.digitize.aperture \\
					When set to 1 or \emph{ON}, this attribute sets the aperture setting to \emph{AUTO} for the selected function.
				\end{tabular} \\
		%---
		DigtzeCount-SP & \begin{tabular}{B}
					long \\
					Min=1 \\
					Max=55000000
				\end{tabular} & 
				\begin{tabular}{P}
					TSP command: dmm.digitize.count \\
					This attribute sets the number of measurements to digitize when a measurement is requested.
				\end{tabular} \\

		DigtzeCount-RB & \begin{tabular}{B}
					long
				\end{tabular} & 
				\begin{tabular}{P}
					TSP command: dmm.digitize.count \\
					This attribute shows the number of measurements to digitize when a measurement is requested.
				\end{tabular} \\ \hline
		%---
		DigtzeSR-SP & \begin{tabular}{B}
					long \\
					Min=1000 \\
					Max=1000000
				\end{tabular} & 
				\begin{tabular}{P}
					TSP command: dmm.digitize.samplerate \\
					This attribute defines the precise acquisition rate at which the digitizing measurements are made.
				\end{tabular} \\

		DigtzeSR-RB & \begin{tabular}{B}
					long
				\end{tabular} & 
				\begin{tabular}{P}
					TSP command: dmm.digitize.samplerate \\
					This attribute shows the precise acquisition rate at which the digitizing measurements are made.
				\end{tabular} \\ \hline
		%---
		DRange-SP & \begin{tabular}{B}
					long \\
					Min=0
				\end{tabular} & 
				\begin{tabular}{P}
					TSP command: dmm.digitize.range \\
					This attribute determines the positive full-scale measure range for digitizer.
				\end{tabular} \\

		DRange-RB & \begin{tabular}{B}
					long
				\end{tabular} & 
				\begin{tabular}{P}
					TSP command: dmm.digitize.range \\
					This attribute shows the positive full-scale measure range for digitizer.
				\end{tabular} \\ \hline
		%---
		DigtzeCoup-Sel & bool\{\begin{itemize}[noitemsep]
					\small
					\item[] AC
					\item[] DC
				\end{itemize}\} & 
				\begin{tabular}{P}
					TSP command: dmm.digitize.coupling.type \\
					This attribute determines if AC or DC signal coupling is used.
				\end{tabular} \\

		DigtzeCoup-Sts & bool\{\begin{itemize}[noitemsep]
					\small
					\item[] AC
					\item[] DC
				\end{itemize}\} & 
				\begin{tabular}{P}
					TSP command: dmm.digitize.coupling.type \\
					This attribute shows if AC or DC signal coupling is used.
				\end{tabular} \\ \hline
		%---
		DigtzeACFilter-Sel & bool\{\begin{itemize}[noitemsep]
					\small
					\item[] Slow
					\item[] Fast
				\end{itemize}\} & 
				\begin{tabular}{P}
					TSP command: dmm.digitize.coupling.acfilter \\
					This attribute selects the instrument settling time when coupling is set to AC.
				\end{tabular} \\

		DigtzeACFilter-Sts & bool\{\begin{itemize}[noitemsep]
					\small
					\item[] Slow
					\item[] Fast
				\end{itemize}\} & 
				\begin{tabular}{P}
					TSP command: dmm.digitize.coupling.acfilter \\
					This attribute shows the instrument settling time when coupling is set to AC.
				\end{tabular} \\ \hline
		%---
		DigtzeACFreq-SP & \begin{tabular}{B}
					float \\
					Min=3\\
					Max=1000000 \\
					unit: \SI{}{\hertz}
				\end{tabular} & 
				\begin{tabular}{P}
					TSP command: dmm.digitize.coupling.acfrequency \\
					This attribute allows you to optimize the amplitude to compensate for signal loss across the coupling capacitor when AC coupling is selected.
				\end{tabular} \\

		DigtzeACFreq-RB & \begin{tabular}{B}
					float
					unit: \SI{}{\hertz}
				\end{tabular} & 
				\begin{tabular}{P}
					TSP command: dmm.digitize.coupling.acfrequency \\
					This attribute shows the AC frequency setting used to optimize the amplitude to compensate for signal loss across the coupling capacitor when AC coupling is selected.
				\end{tabular} \\ \hline
		%---
		DigtzeImpedance-Sel & bool\{\begin{itemize}[noitemsep]
					\small
					\item[] AUTO
					\item[] 10MOhm
				\end{itemize}\} & 
				\begin{tabular}{P}
					TSP command: dmm.digitize.inputimpedance \\
					This attribute determines when the 10 MΩ input divider is enabled.
				\end{tabular} \\

		DigtzeImpedance-Sts & bool\{\begin{itemize}[noitemsep]
					\small
					\item[] AUTO
					\item[] 10MOhm
				\end{itemize}\} & 
				\begin{tabular}{P}
					TSP command: dmm.digitize.inputimpedance \\
					This attribute shows when the 10 MΩ input divider is enabled.
				\end{tabular} \\ \hline
		%---
		DigtzeStim-Sel & enum\{\begin{itemize}[noitemsep]
					\small
					\item[] EVENT\_NONE
					\item[] EVENT\_DISPLAY
					\item[] EVENT\_NOTIFY\textless n\textgreater
					\item[] ($1\leq n\leq 8$)
					\item[] EVENT\_COMMAND
					\item[] EVENT\_DIGIO\textless n\textgreater
					\item[] ($1\leq n\leq 6$)
					\item[] EVENT\_TSPLINK\textless n\textgreater
					\item[] ($1\leq n\leq 3$)
					\item[] EVENT\_LAN\textless n\textgreater
					\item[] ($1\leq n\leq 8$)
					\item[] EVENT\_BLENDER\textless n\textgreater 
					\item[] ($1\leq n\leq 2$)
					\item[] EVENT\_TIMER\textless n\textgreater
					\item[] ($1\leq n\leq 4$)
					\item[] EVENT\_ANALOGTRIGGER
					\item[] EVENT\_EXTERNAL
				\end{itemize}\} & 
				\begin{tabular}{P}
					TSP command: dmm.trigger.digitize.stimulus \\
					This attribute sets the instrument to digitize a measurement when it detects the specified trigger event.
				\end{tabular} \\

		DigtzeStim-Sts & enum\{\begin{itemize}[noitemsep]
					\small
					\item[] EVENT\_NONE
					\item[] EVENT\_DISPLAY
					\item[] EVENT\_NOTIFY\textless n\textgreater
					\item[] ($1\leq n\leq 8$)
					\item[] EVENT\_COMMAND
					\item[] EVENT\_DIGIO\textless n\textgreater
					\item[] ($1\leq n\leq 6$)
					\item[] EVENT\_TSPLINK\textless n\textgreater
					\item[] ($1\leq n\leq 3$)
					\item[] EVENT\_LAN\textless n\textgreater
					\item[] ($1\leq n\leq 8$)
					\item[] EVENT\_BLENDER\textless n\textgreater 
					\item[] ($1\leq n\leq 2$)
					\item[] EVENT\_TIMER\textless n\textgreater
					\item[] ($1\leq n\leq 4$)
					\item[] EVENT\_ANALOGTRIGGER
					\item[] EVENT\_EXTERNAL
				\end{itemize}\} & 
				\begin{tabular}{P}
					TSP command: dmm.trigger.digitize.stimulus \\
					This attribute shows the specified trigger event that causes a measurement digitize.
				\end{tabular} \\ \hline
		%---
		DATrMode-Sel & enum\{\begin{itemize}[noitemsep]
					\small
					\item[] OFF
					\item[] Edge
					\item[] Pulse
					\item[] Window
				\end{itemize}\} & 
				\begin{tabular}{P}
					TSP command: dmm.digitize.analogtrigger.mode \\
					This attribute configures the type of signal behavior that can generate an analog trigger event.
				\end{tabular} \\

		DATrMode-Sts & enum\{\begin{itemize}[noitemsep]
					\small
					\item[] OFF
					\item[] Edge
					\item[] Pulse
					\item[] Window
				\end{itemize}\} & 
				\begin{tabular}{P}
					TSP command: dmm.digitize.analogtrigger.mode \\
					This attribute shows the type of signal behavior that can generate an analog trigger event.
				\end{tabular} \\ \hline
		%---
		DATrEdgeSlp-Sel & bool\{\begin{itemize}[noitemsep]
					\small
					\item[] Rising
					\item[] Falling
				\end{itemize}\} & 
				\begin{tabular}{P}
					TSP command: dmm.digitize.analogtrigger.edge.slope \\
					This attribute defines the slope of the analog trigger edge.
				\end{tabular} \\

		DATrEdgeSlp-Sts & bool\{\begin{itemize}[noitemsep]
					\small
					\item[] Rising
					\item[] Falling
				\end{itemize}\} & 
				\begin{tabular}{P}
					TSP command: dmm.digitize.analogtrigger.edge.slope \\
					This attribute shows the slope of the analog trigger edge.
				\end{tabular} \\ \hline
		%---
		DATrEdgeLvl-SP & \begin{tabular}{B}
					float 
				\end{tabular} & 
				\begin{tabular}{P}
					TSP command: dmm.digitize.analogtrigger.edge.level \\
					This attribute defines the signal level that generates the analog trigger event for the edge trigger mode.
				\end{tabular} \\

		DATrEdgeLvl-RB & \begin{tabular}{B}
					float 
				\end{tabular} & 
				\begin{tabular}{P}
					TSP command: dmm.digitize.analogtrigger.edge.level \\
					This attribute shows the signal level that generates the analog trigger event for the edge trigger mode.
				\end{tabular} \\ \hline
		%---
		DATrHFR-Sel & bool\{\begin{itemize}[noitemsep]
					\small
					\item[] OFF
					\item[] ON
				\end{itemize}\} & 
				\begin{tabular}{P}
					TSP command: dmm.digitize.analogtrigger.highfreqreject \\
					This attribute enables or disables high frequency rejection on analog trigger events.
				\end{tabular} \\

		DATrHFR-Sts & bool\{\begin{itemize}[noitemsep]
					\small
					\item[] OFF
					\item[] ON
				\end{itemize}\} & 
				\begin{tabular}{P}
					TSP command: dmm.digitize.analogtrigger.highfreqreject \\
					This attribute shows if high frequency rejection on analog trigger events is enabled.
				\end{tabular} \\ \hline
		%---
		DATrPulCond-Sel & bool\{\begin{itemize}[noitemsep]
					\small
					\item[] Greater
					\item[] Less
				\end{itemize}\} & 
				\begin{tabular}{P}
					TSP command: dmm.digitize.analogtrigger.pulse.condition \\
					This attribute defines if the pulse must be greater than or less than the pulse width before an analog trigger is generated.
				\end{tabular} \\

		DATrPulCond-Sts & bool\{\begin{itemize}[noitemsep]
					\small
					\item[] Greater
					\item[] Less
				\end{itemize}\} & 
				\begin{tabular}{P}
					TSP command: dmm.digitize.analogtrigger.pulse.condition \\
					This attribute shows if the pulse must be greater than or less than the pulse width before an analog trigger is generated.
				\end{tabular} \\ \hline
		%---
		DATrPulPol-Sel & bool\{\begin{itemize}[noitemsep]
					\small
					\item[] Above
					\item[] Below
				\end{itemize}\} & 
				\begin{tabular}{P}
					TSP command: dmm.digitize.analogtrigger.pulse.polarity \\
					This attribute defines the polarity of the pulse that generates an analog trigger event.
				\end{tabular} \\

		DATrPulPol-Sts & bool\{\begin{itemize}[noitemsep]
					\small
					\item[] Above
					\item[] Below
				\end{itemize}\} & 
				\begin{tabular}{P}
					TSP command: dmm.digitize.analogtrigger.pulse.polarity \\
					This attribute shows the polarity of the pulse that generates an analog trigger event.
				\end{tabular} \\ \hline
		%---
		DATrPulLvl-SP & \begin{tabular}{B}
					float
				\end{tabular} & 
				\begin{tabular}{P}
					TSP command: dmm.digitize.analogtrigger.pulse.level \\
					This attribute defines the pulse level that generates an analog trigger event.
				\end{tabular} \\

		DATrPulLvl-RB & \begin{tabular}{B}
					float
				\end{tabular} & 
				\begin{tabular}{P}
					TSP command: dmm.digitize.analogtrigger.pulse.level \\
					This attribute shows the pulse level that generates an analog trigger event.
				\end{tabular} \\ \hline
		%---
		DATrPulWidth-SP & \begin{tabular}{B}
					float \\
					Min=0.000001 \\
					Max=0.04
				\end{tabular} & 
				\begin{tabular}{P}
					TSP command: dmm.digitize.analogtrigger.pulse.width \\
					This attribute defines the threshold value for the pulse width.
				\end{tabular} \\

		DATrPulWidth-RB & \begin{tabular}{B}
					float
				\end{tabular} & 
				\begin{tabular}{P}
					TSP command: dmm.digitize.analogtrigger.pulse.width \\
					This attribute shows the threshold value for the pulse width.
				\end{tabular} \\ \hline
		%---
		DATrWindHigh-SP & \begin{tabular}{B}
					float
				\end{tabular} & 
				\begin{tabular}{P}
					TSP command: dmm.digitize.analogtrigger.window.levelhigh \\
					This attribute defines the upper boundary of the analog trigger window.
				\end{tabular} \\

		DATrWindHigh-RB & \begin{tabular}{B}
					float
				\end{tabular} & 
				\begin{tabular}{P}
					TSP command: dmm.digitize.analogtrigger.window.levelhigh \\
					This attribute shows the upper boundary of the analog trigger window.
				\end{tabular} \\ \hline
		%---
		DATrWindLow-SP & \begin{tabular}{B}
					float
				\end{tabular} & 
				\begin{tabular}{P}
					TSP command: dmm.digitize.analogtrigger.window.levellow \\
					This attribute defines the lower boundary of the analog trigger window.
				\end{tabular} \\

		DATrWindLow-RB & \begin{tabular}{B}
					float
				\end{tabular} & 
				\begin{tabular}{P}
					TSP command: dmm.digitize.analogtrigger.window.levellow \\
					This attribute shows the lower boundary of the analog trigger window.
				\end{tabular} \\ \hline
		%---
		DATrWindDir-Sel & bool\{\begin{itemize}[noitemsep]
					\small
					\item[] Enter
					\item[] Leave
				\end{itemize}\} & 
				\begin{tabular}{P}
					TSP command: dmm.digitize.analogtrigger.window.direction \\
					This attribute defines if the analog trigger occurs when the signal enters or leaves the defined upper and lower analog signal level boundaries.
				\end{tabular} \\

		DATrWindDir-Sts & bool\{\begin{itemize}[noitemsep]
					\small
					\item[] Enter
					\item[] Leave
				\end{itemize}\} & 
				\begin{tabular}{P}
					TSP command: dmm.digitize.analogtrigger.window.direction \\
					This attribute shows if the analog trigger occurs when the signal enters or leaves the defined upper and lower analog signal level boundaries.
				\end{tabular} \\ \hline
		%---
		DRelOffEnbl-Sel & bool\{\begin{itemize}[noitemsep]
					\small
					\item[] OFF
					\item[] ON
				\end{itemize}\} & 
				\begin{tabular}{P}
					TSP command: dmm.digitize.rel.enable \\
					This attribute enables or disables the application of a relative offset value to the measurement.
				\end{tabular} \\

		DRelOffEnbl-Sts & bool\{\begin{itemize}[noitemsep]
					\small
					\item[] OFF
					\item[] ON
				\end{itemize}\} & 
				\begin{tabular}{P}
					TSP command: dmm.digitize.rel.enable \\
					This attribute shows if the application of a relative offset value to the measurement is enabled.
				\end{tabular} \\ \hline
		%---
		DRelOffAcq-Cmd & bool\{\begin{itemize}[noitemsep]
					\small
					\item[] OFF
					\item[] ON
				\end{itemize}\} & 
				\begin{tabular}{P}
					TSP command: dmm.digitize.rel.acquire() \\
					When set to 1 or \emph{ON}, this function acquires a measurement and stores it as the relative offset value.
				\end{tabular} \\ \hline
		%---
		DRelOff-SP & \begin{tabular}{B}
					float 
				\end{tabular} & 
				\begin{tabular}{P}
					TSP command: dmm.digitize.rel.level \\
					This attribute sets the relative offset value.
				\end{tabular} \\

		DRelOff-RB & \begin{tabular}{B}
					float 
				\end{tabular} & 
				\begin{tabular}{P}
					TSP command: dmm.digitize.rel.level \\
					This attribute shows the relative offset value.
				\end{tabular} \\ \hline
		%---
		DMathEnbl-Sel & bool\{\begin{itemize}[noitemsep]
					\small
					\item[] OFF
					\item[] ON
				\end{itemize}\} & 
				\begin{tabular}{P}
					TSP command: dmm.digitize.math.enable \\
					This attribute enables or disables math operations on measurements for the selected digitize function.
				\end{tabular} \\

		DMathEnbl-Sts & bool\{\begin{itemize}[noitemsep]
					\small
					\item[] OFF
					\item[] ON
				\end{itemize}\} & 
				\begin{tabular}{P}
					TSP command: dmm.digitize.math.enable \\
					This attribute enables or disables math operations on measurements for the selected digitize function.
				\end{tabular} \\ \hline
		%---
		DMathOp-Sel & enum\{\begin{itemize}[noitemsep]
					\small
					\item[] y=mx+b
					\item[] Percent
					\item[] Reciprocal
				\end{itemize}\} & 
				\begin{tabular}{P}
					TSP command: dmm.digitize.math.format \\
					This attribute specifies which math operation is performed on measurements when math operations are enabled.
				\end{tabular} \\

		DMathOp-Sts & bool\{\begin{itemize}[noitemsep]
					\small
					\item[] y=mx+b
					\item[] Percent
					\item[] Reciprocal
				\end{itemize}\} & 
				\begin{tabular}{P}
					TSP command: dmm.digitize.math.format \\
					This attribute shows which math operation is performed on measurements when math operations are enabled.
				\end{tabular} \\ \hline
		%---
		DMathBFactor-SP & \begin{tabular}{B}
					float \\
					Min=-1000000000000 \\
					Max=1000000000000
				\end{tabular} & 
				\begin{tabular}{P}
					TSP command: dmm.digitize.math.mxb.bfactor \\
					This attribute specifies the offset, b, for the y = mx + b operation.
				\end{tabular} \\

		DMathBFactor-RB & \begin{tabular}{B}
					float
				\end{tabular} & 
				\begin{tabular}{P}
					TSP command: dmm.digitize.math.mxb.bfactor \\
					This attribute shows the offset, b, for the y = mx + b operation.
				\end{tabular} \\ \hline
		%---
		DMathMFactor-SP & \begin{tabular}{B}
					float \\
					Min=-1000000000000 \\
					Max=1000000000000
				\end{tabular} & 
				\begin{tabular}{P}
					TSP command: dmm.digitize.math.mxb.mfactor \\
					This attribute specifies the scale factor, m, for the y = mx + b math operation.
				\end{tabular} \\

		DMathMFactor-RB & \begin{tabular}{B}
					float
				\end{tabular} & 
				\begin{tabular}{P}
					TSP command: dmm.digitize.math.mxb.mfactor \\
					This attribute shows the scale factor, m, for the y = mx + b math operation.
				\end{tabular} \\ \hline
		%---
		DMathPercRef-SP & \begin{tabular}{B}
					float \\
					Min=-1000000000000 \\
					Max=1000000000000
				\end{tabular} & 
				\begin{tabular}{P}
					TSP command: dmm.digitize.math.percent \\
					This attribute specifies the reference constant that is used when math operations are set to percent.
				\end{tabular} \\

		DMathPercRef-RB & \begin{tabular}{B}
					float
				\end{tabular} & 
				\begin{tabular}{P}
					TSP command: dmm.digitize.math.percent \\
					This attribute shows the reference constant that is used when math operations are set to percent.
				\end{tabular} \\ \hline
	\end{longtable}

	% TABLE: Buffer operations
	\begin{longtable}{A B P}
		\caption{Buffer operations} \\ \hline
		\bfseries Name & \bfseries Data Type & \bfseries Description \\ \hline
		%---
		StartRead\textless n\textgreater-SP & \begin{tabular}{B}
					long \\
					Min=1 
				\end{tabular} & 
				\begin{tabular}{P}
					TSP command: No command \\
					n: number of default buffer (1 or 2) \\
					This attribute specifies the start index to start reading the buffer when a read buffer command is issued.
				\end{tabular} \\ \hline
		%---
		EndRead\textless n\textgreater-SP & \begin{tabular}{B}
					long \\
					Min=1 
				\end{tabular} & 
				\begin{tabular}{P}
					TSP command: No command \\
					n: number of default buffer (1 or 2) \\
					This attribute specifies the last buffer index to read from when a read buffer command is issued.
				\end{tabular} \\ \hline
		%---
		ReadBuff\textless n\textgreater-Cmd & bool\{\begin{itemize}[noitemsep]
					\small
					\item[] OFF
					\item[] ON
				\end{itemize}\} & 
				\begin{tabular}{P}
					TSP command: No command \\
					n: number of default buffer (1 or 2) \\
					When set to 1 or emph{ON}, this command causes ReadBuff\textless n\textgreater-Mon to process once to read the buffer section specified by \emph{StartRead\textless n\textgreater-SP} and \emph{EndRead\textless n\textgreater-SP}.
				\end{tabular} \\ \hline
		%---
		ReadBuff\textless n\textgreater-Mon & \begin{tabular}{B}
					double[1000]
				\end{tabular} & 
				\begin{tabular}{P}
					TSP command: printbuffer(StartRead\textless n\textgreater, EndRead\textless n\textgreater, defbuffer\textless n\textgreater.readings) \\
					n: number of default buffer (1 or 2) \\
					This variable reads an array of readings from the corresponding buffer when processed.  Set \emph{ReadBuff\textless n\textgreater-Mon.PROC} to any value to retrieve buffer readings once. Set \emph{ReadBuff\textless n\textgreater-Mon.SCAN} to an EPICS SCAN valide value in order to get data from the buffer periodically.
				\end{tabular} \\ \hline
		%---
		FetchBuff\textless n\textgreater-Mon & \begin{tabular}{B}
					double
				\end{tabular} & 
				\begin{tabular}{P}
					TSP command: printbuffer(defbuffer\textless n\textgreater.endindex, defbuffer\textless n\textgreater.endindex, defbuffer\textless n\textgreater.readings) \\
					n: number of default buffer (1 or 2) \\
					This variable reads the latest measurement from the corresponding buffer when processed. Set \emph{FetchBuff\textless n\textgreater-Mon.PROC} to any value in order to fetch a measurement once. Set \emph{FetchBuff\textless n\textgreater-Mon.SCAN} to an EPICS SCAN valide value in order to fetch measurements periodically.
				\end{tabular} \\ \hline
		%---
		MeasBuff\textless n\textgreater-Cmd & bool\{\begin{itemize}[noitemsep]
					\small
					\item[] OFF
					\item[] ON
				\end{itemize}\} & 
				\begin{tabular}{P}
					TSP command: print(dmm.measure.read(defbuffer\textless n\textgreater)) \\
					n: number of default buffer (1 or 2) \\
					When set to 1 or \emph{ON}, this command causes a measurement to be taken (a measure function must be selected) and stored in the \emph{FetchBuff\textless n\textgreater-Mon} variable.
				\end{tabular} \\ \hline
		%---
		DigtzBuff\textless n\textgreater-Cmd & bool\{\begin{itemize}[noitemsep]
					\small
					\item[] OFF
					\item[] ON
				\end{itemize}\} & 
				\begin{tabular}{P}
					TSP command: print(dmm.digitize.read(defbuffer\textless n\textgreater)) \\
					n: number of default buffer (1 or 2) \\
					When set to 1 or \emph{ON}, this command causes a digitize measurement to be taken (a digitize function must be selected) and stored in the \emph{FetchBuff\textless n\textgreater-Mon} variable.
				\end{tabular} \\ \hline
		%---
		StartBuff\textless n\textgreater-Mon & \begin{tabular}{B}
					long
				\end{tabular} & 
				\begin{tabular}{P}
					TSP command: print(defbuffer\textless n\textgreater.startindex) \\
					n: number of default buffer (1 or 2) \\
					This variable contains the corresponding buffer start index. Set \emph{StartBuff\textless n\textgreater-Mon.PROC} to any value in order to read the buffer start index once. Set \emph{StartBuff\textless n\textgreater-Mon.SCAN} to an EPICS SCAN valide value in order to monitor the buffer start index, periodically.
				\end{tabular} \\ \hline
		%---
		EndBuff\textless n\textgreater-Mon & \begin{tabular}{B}
					long
				\end{tabular} & 
				\begin{tabular}{P}
					TSP command: print(defbuffer\textless n\textgreater.endindex) \\
					n: number of default buffer (1 or 2) \\
					This variable contains the corresponding buffer end index. Set \emph{EndBuff\textless n\textgreater-Mon.PROC} to any value in order to read the buffer end index once. Set \emph{EndBuff\textless n\textgreater-Mon.SCAN} to an EPICS SCAN valide value in order to monitor the buffer end index, periodically.
				\end{tabular} \\ \hline
		%---
		SizeBuff\textless n\textgreater-SP & \begin{tabular}{B}
					long \\
					Min=0
				\end{tabular} & 
				\begin{tabular}{P}
					TSP command: defbuffer\textless n\textgreater.capacity \\
					n: number of default buffer (1 or 2) \\
					This attribute specifies the number of readings the buffer can store.
				\end{tabular} \\

		SizeBuff\textless n\textgreater-RB & \begin{tabular}{B}
					long
				\end{tabular} & 
				\begin{tabular}{P}
					TSP command: defbuffer\textless n\textgreater.capacity \\
					n: number of default buffer (1 or 2) \\
					This attribute shows the number of readings the buffer can store.
				\end{tabular} \\ \hline
		%---
		FillModeBuff\textless n\textgreater-Sel & bool\{\begin{itemize}[noitemsep]
					\small
					\item[] Once
					\item[] Continuous
				\end{itemize}\} & 
				\begin{tabular}{P}
					TSP command: defbuffer\textless n\textgreater.fillmode \\
					n: number of default buffer (1 or 2) \\
					This attribute determines if a reading buffer is filled continuously or is filled once and stops.
				\end{tabular} \\

		FillModeBuff\textless n\textgreater-Sts & bool\{\begin{itemize}[noitemsep]
					\small
					\item[] Once
					\item[] Continuous
				\end{itemize}\} & 
				\begin{tabular}{P}
					TSP command: defbuffer\textless n\textgreater.fillmode \\
					n: number of default buffer (1 or 2) \\
					This attribute shows if a reading buffer is filled continuously or is filled once and stops.
				\end{tabular} \\ \hline
		%---
		CntBuff\textless n\textgreater-Mon & \begin{tabular}{B}
					long
				\end{tabular} & 
				\begin{tabular}{P}
					TSP command: defbuffer\textless n\textgreater.n \\
					n: number of default buffer (1 or 2) \\
					This variable gets the number of readings in the specified reading buffer when processed. Set \emph{CntBuff\textless n\textgreater-Mon.PROC} to any value in order to read the buffer count once. Set \emph{CntBuff\textless n\textgreater-Mon.SCAN} to an EPICS SCAN valide value in order to periodically monitor the buffer count.
				\end{tabular} \\ \hline
		%---
		ClrBuff\textless n\textgreater-Cmd & bool\{\begin{itemize}[noitemsep]
					\small
					\item[] OFF
					\item[] ON
				\end{itemize}\} & 
				\begin{tabular}{P}
					TSP command: defbuffer\textless n\textgreater.clear() \\
					n: number of default buffer (1 or 2) \\
					When set to 1 or \emph{ON}, this command clears all readings and statistics from the buffer.
				\end{tabular} \\ \hline
		%---
		AvgBuff\textless n\textgreater-Mon & \begin{tabular}{B}
					float
				\end{tabular} & 
				\begin{tabular}{P}
					TSP command: statsVar = buffer.getstats(defbuffer\textless n\textgreater-Mon); print(statsVar.mean); \\
					n: number of default buffer (1 or 2) \\
					This variable gets the average of the corresponding buffer readings when processed. Set \emph{AvgBuff\textless n\textgreater-Mon.PROC} to any value in order to read the buffer average value once. Set \emph{AvgBuff\textless n\textgreater-Mon.SCAN} to an EPICS SCAN valide value in order to monitor the buffer average periodically.
				\end{tabular} \\ \hline
		%---
		MaxBuff\textless n\textgreater-Mon & \begin{tabular}{B}
					float
				\end{tabular} & 
				\begin{tabular}{P}
					TSP command: statsVar = buffer.getstats(defbuffer\textless n\textgreater); print(statsVar.max.reading; \\
					n: number of default buffer (1 or 2) \\
					This variable gets the maximum value reading from the corresponding buffer when processed. Set \emph{MaxBuff\textless n\textgreater-Mon.PROC} to any value in order to read the buffer maximum value once. Set \emph{MaxBuff\textless n\textgreater-Mon.SCAN} to an EPICS SCAN valide value in order to monitor the buffer maximum value periodically.
				\end{tabular} \\ \hline
		%---
		MinBuff\textless n\textgreater-Mon & \begin{tabular}{B}
					float
				\end{tabular} & 
				\begin{tabular}{P}
					TSP command: statsVar = buffer.getstats(defbuffer\textless n\textgreater); print(statsVar.min.reading); \\
					n: number of default buffer (1 or 2) \\
					This variable gets the minimum value reading from the corresponding buffer when processed. Set \emph{MinBuff\textless n\textgreater-Mon.PROC} to any value in order to read the buffer maximum value once. Set \emph{MinBuff\textless n\textgreater-Mon.SCAN} to an EPICS SCAN valide value in order to monitor the buffer minimum value periodically.
				\end{tabular} \\

		StdDBuff\textless n\textgreater-Mon & \begin{tabular}{B}
					float
				\end{tabular} & 
				\begin{tabular}{P}
					TSP command: statsVar = buffer.getstats(defbuffer\textless n\textgreater); print(statsVar.stddev); \\
					n: number of default buffer (1 or 2) \\
					This variable gets the standard deviation of readings for the corresponding buffer when processed. Set \emph{StdDBuff\textless n\textgreater-Mon.PROC} to any value in order to read the buffer standard devitaion once. Set \emph{StdDBuff\textless n\textgreater-Mon.SCAN} to an EPICS SCAN valide value in order to monitor the buffer standard deviation periodically.
				\end{tabular} \\ \hline
		%---
		ClrStatBuff\textless n\textgreater-Cmd & bool\{\begin{itemize}[noitemsep]
					\small
					\item[] OFF
					\item[] ON
				\end{itemize}\} & 
				\begin{tabular}{P}
					TSP command: buffer.clearstats(defbuffer\textless n\textgreater) \\
					n: number of default buffer (1 or 2) \\
					When set to 1 or \emph{ON}, this command clears the statistical information associated with the specified buffer.
				\end{tabular} \\
	\end{longtable}
	% TABLE: External I/O
	\begin{longtable}{A B P}
		\caption{External I/O} \\ \hline
		\bfseries Name & \bfseries Data Type & \bfseries Description \\ \hline
		%---
		ExInEdge-Sel & enum\{\begin{itemize}[noitemsep]
					\small
					\item[] FALLING
					\item[] RISING
					\item[] EITHER
				\end{itemize}\} & 
				\begin{tabular}{P}
					TSP command: trigger.extin.edge \\
					This attribute sets the type of edge that is detected as an input on the external in line.
				\end{tabular} \\ \hline
		%---
		ExInEdge-Sts & bool\{\begin{itemize}[noitemsep]
					\small
					\item[] FALLING
					\item[] RISING
					\item[] EITHER
				\end{itemize}\} & 
				\begin{tabular}{P}
					TSP command: trigger.extin.edge \\
					This attribute shows the type of edge that is detected as an input on the external in line.
				\end{tabular} \\ \hline
		%---
		ExInOver-Mon & bool\{\begin{itemize}[noitemsep]
					\small
					\item[] No overrun
					\item[] Overrun
				\end{itemize}\} & 
				\begin{tabular}{P}
					TSP command: trigger.extin.overrun \\
					This attribute shows the event detector overrun status.
				\end{tabular} \\ \hline
		%---
		ClearExInEv-Cmd & bool\{\begin{itemize}[noitemsep]
					\small
					\item[] OFF
					\item[] ON
				\end{itemize}\} & 
				\begin{tabular}{P}
					TSP command: trigger.extin.clear() \\
					When set to 1 or \emph{ON}, this command clears the trigger event on the external in line.
				\end{tabular} \\ \hline
		%---
		ExOutPol-Sel & bool\{\begin{itemize}[noitemsep]
					\small
					\item[] Positive
					\item[] Negative
				\end{itemize}\} & 
				\begin{tabular}{P}
					TSP command: trigger.extout.logic \\
					This attribute sets the output logic of the trigger event generator to positive or negative for the external out line.
				\end{tabular} \\

		ExOutPol-Sts & bool\{\begin{itemize}[noitemsep]
					\small
					\item[] Positive
					\item[] Negative
				\end{itemize}\} & 
				\begin{tabular}{P}
					TSP command: trigger.extout.logic \\
					This attribute shows the output logic of the trigger event generator to positive or negative for the external out line.
				\end{tabular} \\ \hline
		%---
		ExOutStim-Sel & bool\{\begin{itemize}[noitemsep]
					\small
					\item[] EVENT\_NONE
					\item[] EVENT\_DISPLAY
					\item[] EVENT\_NOTIFY\textless n\textgreater
					\item[] ($1\leq n\leq 8$)
					\item[] EVENT\_COMMAND
					\item[] EVENT\_DIGIO\textless n\textgreater
					\item[] ($1\leq n\leq 6$)
					\item[] EVENT\_TSPLINK\textless n\textgreater
					\item[] ($1\leq n\leq 3$)
					\item[] EVENT\_LAN\textless n\textgreater
					\item[] ($1\leq n\leq 8$)
					\item[] EVENT\_BLENDER\textless n\textgreater 
					\item[] ($1\leq n\leq 2$)
					\item[] EVENT\_TIMER\textless n\textgreater
					\item[] ($1\leq n\leq 4$)
					\item[] EVENT\_ANALOGTRIGGER
					\item[] EVENT\_EXTERNAL
				\end{itemize}\} & 
				\begin{tabular}{P}
					TSP command: trigger.extout.stimulus \\
					This attribute selects the event that causes a trigger to be asserted on the external output line.
				\end{tabular} \\

		ExOutStim-Sts & bool\{\begin{itemize}[noitemsep]
					\small
					\item[] EVENT\_NONE
					\item[] EVENT\_DISPLAY
					\item[] EVENT\_NOTIFY\textless n\textgreater
					\item[] ($1\leq n\leq 8$)
					\item[] EVENT\_COMMAND
					\item[] EVENT\_DIGIO\textless n\textgreater
					\item[] ($1\leq n\leq 6$)
					\item[] EVENT\_TSPLINK\textless n\textgreater
					\item[] ($1\leq n\leq 3$)
					\item[] EVENT\_LAN\textless n\textgreater
					\item[] ($1\leq n\leq 8$)
					\item[] EVENT\_BLENDER\textless n\textgreater 
					\item[] ($1\leq n\leq 2$)
					\item[] EVENT\_TIMER\textless n\textgreater
					\item[] ($1\leq n\leq 4$)
					\item[] EVENT\_ANALOGTRIGGER
					\item[] EVENT\_EXTERNAL
				\end{itemize}\} & 
				\begin{tabular}{P}
					TSP command: trigger.extout.stimulus \\
					This attribute shows the event that causes a trigger to be asserted on the external output line.
				\end{tabular} \\ \hline
	\end{longtable}
	% TABLE: Digital I/O
	\begin{longtable}{A B P}
		\caption{Digital I/O} \\ \hline
		\bfseries Name & \bfseries Data Type & \bfseries Description \\ \hline
		%---
		DigWrite-SP & \begin{tabular}{B}
					long \\
					Min=0 \\ 
					Max=63
				\end{tabular} & 
				\begin{tabular}{P}
					TSP command: digio.writeport() \\
					This variable writes to all digital I/O lines.
				\end{tabular} \\ \hline
		%---
		DigRead-Mon & \begin{tabular}{B}
					long
				\end{tabular} & 
				\begin{tabular}{P}
					TSP command: digio.readport() \\
					This variable reads the digital I/O port (all lines) when processed. Set \emph{DigRead-Mon.PROC} to any value in order to read the I/O port once. Set \emph{DigRead-Mon.SCAN} to an EPICS SCAN valide value in order to monitor the port value periodically.
				\end{tabular} \\ \hline
		%---
		Dig{\textless n\textgreater}Mod-Sel & enum\{\begin{itemize}[noitemsep]
					\small
					\item[] DIGIN
					\item[] DIGOUT
					\item[] DIGOPEN
					\item[] TRIGIN
					\item[] TRIGOUT
					\item[] TRIGOPEN
					\item[] SYNCMASTER
					\item[] SYNCACC
				\end{itemize}\} & 
				\begin{tabular}{P}
					TSP command: digio.line[n].mode \\
					n: digital line number (1 to 6) \\
					This attribute sets the mode of the digital I/O line to be a digital line, trigger line, or synchronous line and sets the line to be input, output, or open-drain.
				\end{tabular} \\

		Dig{\textless n\textgreater}Mod-Sts & enum\{\begin{itemize}[noitemsep]
					\small
					\item[] DIGIN
					\item[] DIGOUT
					\item[] DIGOPEN
					\item[] TRIGIN
					\item[] TRIGOUT
					\item[] TRIGOPEN
					\item[] SYNCMASTER
					\item[] SYNCACC
				\end{itemize}\} & 
				\begin{tabular}{P}
					TSP command: digio.line[n].mode \\
					n: digital line number (1 to 6) \\
					This attribute shows the mode of the digital I/O line: digital line, trigger line, or synchronous line; and I/O configuration: input, output, or open-drain.
				\end{tabular} \\ \hline
		%---
		Dig{\textless n\textgreater}State-Sel & bool\{\begin{itemize}[noitemsep]
					\small
					\item[] LOW
					\item[] HIGH
				\end{itemize}\} & 
				\begin{tabular}{P}
					TSP command: digio.line[n].state \\
					n: digital line number (1 to 6) \\
					This variable sets the corresponding digital I/O line high or low when the line is set for digital control.
				\end{tabular} \\

		Dig{\textless n\textgreater}State-Mon & bool\{\begin{itemize}[noitemsep]
					\small
					\item[] LOW
					\item[] HIGH
				\end{itemize}\} & 
				\begin{tabular}{P}
					TSP command: digio.line[n].state \\
					n: digital line number (1 to 6) \\
					This variable reads the state of the corresponding digital I/O line.
				\end{tabular} \\ \hline
		%---
		Dig{\textless n\textgreater}ClrEv-Cmd & bool\{\begin{itemize}[noitemsep]
					\small
					\item[] OFF
					\item[] ON
				\end{itemize}\} & 
				\begin{tabular}{P}
					TSP command: trigger.digin[n].clear() \\
					n: digital line number (1 to 6) \\
					When set to 1 or \emph{ON}, this command clears the trigger event on the corresponding digital input line.
				\end{tabular} \\
		%---
		Dig{\textless n\textgreater}Edge-Sel & enum\{\begin{itemize}[noitemsep]
					\small
					\item[] Falling
					\item[] Rising
					\item[] Either
				\end{itemize}\} & 
				\begin{tabular}{P}
					TSP command: trigger.digin[n].edge \\
					n: digital line number (1 to 6) \\
					This attribute sets the edge used by the trigger event detector on the given trigger line.
				\end{tabular} \\

		Dig{\textless n\textgreater}Edge-Sts & enum\{\begin{itemize}[noitemsep]
					\small
					\item[] Falling
					\item[] Rising
					\item[] Either
				\end{itemize}\} & 
				\begin{tabular}{P}
					TSP command: trigger.digin[n].edge \\
					n: digital line number (1 to 6) \\
					This attribute shows the edge used by the trigger event detector on the given trigger line.
				\end{tabular} \\ \hline
		%---
		Dig{\textless n\textgreater}Over-Mon & bool\{\begin{itemize}[noitemsep]
					\small
					\item[] No overrun
					\item[] Overrun
				\end{itemize}\} & 
				\begin{tabular}{P}
					TSP command: trigger.digin[n].overrun \\
					n: digital line number (1 to 6) \\
					This attribute shows the event detector overrun status.
				\end{tabular} \\ \hline
		%---
		Dig{\textless n\textgreater}Pol-Sel & bool\{\begin{itemize}[noitemsep]
					\small
					\item[] Positive
					\item[] Negative
				\end{itemize}\} & 
				\begin{tabular}{P}
					TSP command: trigger.digout[n].logic \\
					n: digital line number (1 to 6) \\
					This attribute sets the output logic of the trigger event generator to positive or negative for the corresponding line.
				\end{tabular} \\

		Dig{\textless n\textgreater}Pol-Sts & bool\{\begin{itemize}[noitemsep]
					\small
					\item[] Positive
					\item[] Negative
				\end{itemize}\} & 
				\begin{tabular}{P}
					TSP command: trigger.digout[n].logic \\
					n: digital line number (1 to 6) \\
					This attribute shows the output logic of the trigger event generator for the corresponding line.
				\end{tabular} \\ \hline
		%---
		Dig{\textless n\textgreater}Width-SP & \begin{tabular}{B}
					float
					Min=0 \\
					Max=100000 
				\end{tabular} & 
				\begin{tabular}{P}
					TSP command: trigger.digout[n].pulsewidth \\
					n: digital line number (1 to 6) \\
					This attribute sets the length of time that the trigger line is asserted for output triggers.
				\end{tabular} \\

		Dig{\textless n\textgreater}Width-RB & \begin{tabular}{B}
					float
				\end{tabular} & 
				\begin{tabular}{P}
					TSP command: trigger.digout[n].pulsewidth \\
					n: digital line number (1 to 6) \\
					This attribute shows the length of time that the trigger line is asserted for output triggers.
				\end{tabular} \\ \hline
		%---
		Dig{\textless n\textgreater}Stim-Sel & enum\{\begin{itemize}[noitemsep]
					\small
					\item[] EVENT\_NONE
					\item[] EVENT\_DISPLAY
					\item[] EVENT\_NOTIFY\textless m\textgreater
					\item[] ($1\leq m\leq 8$)
					\item[] EVENT\_COMMAND
					\item[] EVENT\_DIGIO\textless m\textgreater
					\item[] ($1\leq m\leq 6$)
					\item[] EVENT\_TSPLINK\textless m\textgreater
					\item[] ($1\leq m\leq 3$)
					\item[] EVENT\_LAN\textless m\textgreater
					\item[] ($1\leq m\leq 8$)
					\item[] EVENT\_BLENDER\textless m\textgreater 
					\item[] ($1\leq m\leq 2$)
					\item[] EVENT\_TIMER\textless m\textgreater
					\item[] ($1\leq m\leq 4$)
					\item[] EVENT\_ANALOGTRIGGER
					\item[] EVENT\_EXTERNAL
				\end{itemize}\} & 
				\begin{tabular}{P}
					TSP command: trigger.digout[n].stimulus \\
					n: digital line number (1 to 6) \\
					This attribute selects the event that causes a trigger to be asserted on the corresponding digital output line.
				\end{tabular} \\

		Dig{\textless n\textgreater}Stim-Sts & enum\{\begin{itemize}[noitemsep]
					\small
					\item[] EVENT\_NONE
					\item[] EVENT\_DISPLAY
					\item[] EVENT\_NOTIFY\textless m\textgreater
					\item[] ($1\leq m\leq 8$)
					\item[] EVENT\_COMMAND
					\item[] EVENT\_DIGIO\textless m\textgreater
					\item[] ($1\leq m\leq 6$)
					\item[] EVENT\_TSPLINK\textless m\textgreater
					\item[] ($1\leq m\leq 3$)
					\item[] EVENT\_LAN\textless m\textgreater
					\item[] ($1\leq m\leq 8$)
					\item[] EVENT\_BLENDER\textless m\textgreater 
					\item[] ($1\leq m\leq 2$)
					\item[] EVENT\_TIMER\textless m\textgreater
					\item[] ($1\leq m\leq 4$)
					\item[] EVENT\_ANALOGTRIGGER
					\item[] EVENT\_EXTERNAL
				\end{itemize}\} & 
				\begin{tabular}{P}
					TSP command: trigger.digout[n].stimulus \\
					n: digital line number (1 to 6) \\
					This attribute shows the event that causes a trigger to be asserted on the corresponding digital output line.
				\end{tabular} \\ \hline
		%---
		Dig{\textless n\textgreater}Assert-Cmd & bool\{\begin{itemize}[noitemsep]
					\small
					\item[] OFF
					\item[] ON
				\end{itemize}\} & 
				\begin{tabular}{P}
					TSP command: trigger.digout[n].assert() \\
					n: digital line number (1 to 6) \\
					This function asserts a trigger pulse on the corresponding digital I/O line.
				\end{tabular} \\ \hline
		%---
		Dig{\textless n\textgreater}Release-Cmd & bool\{\begin{itemize}[noitemsep]
					\small
					\item[] OFF
					\item[] ON
				\end{itemize}\} & 
				\begin{tabular}{P}
					TSP command: trigger.digout[n].release() \\
					n: digital line number (1 to 6) \\
					When set to 1 or \emph{ON}, this command releases an indefinite length or latched trigger.
				\end{tabular} \\ \hline
	\end{longtable}
	% TABLE: Timer
	\begin{longtable}{A B P}
		\caption{Timer} \\ \hline
		\bfseries Name & \bfseries Data Type & \bfseries Description \\ \hline
		%---
		Timer{\textless n\textgreater}Enbl-Sel & bool\{\begin{itemize}[noitemsep]
					\small
					\item[] OFF
					\item[] ON
				\end{itemize}\} & 
				\begin{tabular}{P}
					TSP command: trigger.timer[n].enable \\
					n: timer number (1 to 4) \\
					This attribute enables the trigger timer.
				\end{tabular} \\

		Timer{\textless n\textgreater}Enbl-Sts & bool\{\begin{itemize}[noitemsep]
					\small
					\item[] OFF
					\item[] ON
				\end{itemize}\} & 
				\begin{tabular}{P}
					TSP command: trigger.timer[n].enable \\
					n: timer number (1 to 4) \\
					This attribute shows if the trigger timer is enabled.
				\end{tabular} \\ \hline
		%---
		Timer{\textless n\textgreater}Dly-SP & \begin{tabular}{B}
					float \\
					Min=0.000008 \\
					Max=100000
				\end{tabular} & 
				\begin{tabular}{P}
					TSP command: trigger.timer[n].delay \\
					n: timer number (1 to 4) \\
					This attribute sets the timer delay.
				\end{tabular} \\

		Timer{\textless n\textgreater}Dly-RB & \begin{tabular}{B}
					float
				\end{tabular} & 
				\begin{tabular}{P}
					TSP command: trigger.timer[n].delay \\
					n: timer number (1 to 4) \\
					This attribute shows the timer delay. 
				\end{tabular} \\ \hline
		%---
		Timer{\textless n\textgreater}Count-SP & \begin{tabular}{B}
					float \\
					Min=0 \\
					Max=1048575
				\end{tabular} & 
				\begin{tabular}{P}
					TSP command: trigger.timer[n].count \\
					n: timer number (1 to 4) \\
					This attribute sets the number of events to generate each time the timer generates a trigger event or is enabled as a timer or alarm.
				\end{tabular} \\

		Timer{\textless n\textgreater}Count-RB & \begin{tabular}{B}
					float
				\end{tabular} & 
				\begin{tabular}{P}
					TSP command: trigger.timer[n].count \\
					n: timer number (1 to 4) \\
					This attribute shows the number of events that are generated each time the timer triggers an event or is enabled as a timer or alarm.
				\end{tabular} \\ \hline
		%---
		Timer{\textless n\textgreater}Gen-Sel & bool\{\begin{itemize}[noitemsep]
					\small
					\item[] Elapse
					\item[] Start and elapse
				\end{itemize}\} & 
				\begin{tabular}{P}
					TSP command: trigger.timer[n].start.generate \\
					n: timer number (1 to 4) \\
					This attribute specifies when timer events are generated.
				\end{tabular} \\

		Timer{\textless n\textgreater}Gen-Sts & bool\{\begin{itemize}[noitemsep]
					\small
					\item[] Elapse
					\item[] Start and elapse
				\end{itemize}\} & 
				\begin{tabular}{P}
					TSP command: trigger.timer[n].start.generate \\
					n: timer number (1 to 4) \\
					This attribute shows when timer events are generated.
				\end{tabular} \\ \hline
		%---
		Timer{\textless n\textgreater}Sec-SP & \begin{tabular}{B}
					long \\
					Min=0 \\ 
					Max=2147483647
				\end{tabular} & 
				\begin{tabular}{P}
					TSP command: trigger.timer[n].start.seconds \\
					n: timer number (1 to 4) \\
					This attribute configures the seconds of an alarm or a time in the future when the timer will start.
				\end{tabular} \\

		Timer{\textless n\textgreater}Sec-RB & \begin{tabular}{B}
					long
				\end{tabular} & 
				\begin{tabular}{P}
					TSP command: trigger.timer[n].start.seconds \\
					n: timer number (1 to 4) \\
					This attribute shows, in seconds, the time of an alarm or a time in the future when the timer will start.
				\end{tabular} \\ \hline
		%---
		Timer{\textless n\textgreater}Frac-SP & \begin{tabular}{B}
					float \\
					Min=0 \\
					Max=1
				\end{tabular} & 
				\begin{tabular}{P}
					TSP command: trigger.timer[n].start.fractionalseconds \\
					n: timer number (1 to 4) \\
					This attribute configures the fractional seconds of an alarm or a time in the future when the timer will start.
				\end{tabular} \\

		Timer{\textless n\textgreater}Frac-RB & \begin{tabular}{B}
					float
				\end{tabular} & 
				\begin{tabular}{P}
					TSP command: trigger.timer[n].start.fractionalseconds \\
					n: timer number (1 to 4) \\
					This attribute shows the fractional seconds of an alarm or a time in the future when the timer will start.
				\end{tabular} \\ \hline
		%---
		Timer{\textless n\textgreater}Stim-Sel & enum\{\begin{itemize}[noitemsep]
					\small
					\item[] EVENT\_NONE
					\item[] EVENT\_DISPLAY
					\item[] EVENT\_NOTIFY\textless m\textgreater
					\item[] ($1\leq m\leq 8$)
					\item[] EVENT\_COMMAND
					\item[] EVENT\_DIGIO\textless m\textgreater
					\item[] ($1\leq m\leq 6$)
					\item[] EVENT\_TSPLINK\textless m\textgreater
					\item[] ($1\leq m\leq 3$)
					\item[] EVENT\_LAN\textless m\textgreater
					\item[] ($1\leq m\leq 8$)
					\item[] EVENT\_BLENDER\textless m\textgreater 
					\item[] ($1\leq m\leq 2$)
					\item[] EVENT\_TIMER\textless m\textgreater
					\item[] ($1\leq m\leq 4$)
					\item[] EVENT\_ANALOGTRIGGER
					\item[] EVENT\_EXTERNAL
				\end{itemize}\} & 
				\begin{tabular}{P}
					TSP command: trigger.timer[n].start.stimulus \\
					n: timer number (1 to 4) \\
					This attribute sets the event that starts the trigger timer.
				\end{tabular} \\

		Timer{\textless n\textgreater}Stim-Sts & enum\{\begin{itemize}[noitemsep]
					\small
					\item[] EVENT\_NONE
					\item[] EVENT\_DISPLAY
					\item[] EVENT\_NOTIFY\textless m\textgreater
					\item[] ($1\leq m\leq 8$)
					\item[] EVENT\_COMMAND
					\item[] EVENT\_DIGIO\textless m\textgreater
					\item[] ($1\leq m\leq 6$)
					\item[] EVENT\_TSPLINK\textless m\textgreater
					\item[] ($1\leq m\leq 3$)
					\item[] EVENT\_LAN\textless m\textgreater
					\item[] ($1\leq m\leq 8$)
					\item[] EVENT\_BLENDER\textless m\textgreater 
					\item[] ($1\leq m\leq 2$)
					\item[] EVENT\_TIMER\textless m\textgreater
					\item[] ($1\leq m\leq 4$)
					\item[] EVENT\_ANALOGTRIGGER
					\item[] EVENT\_EXTERNAL
				\end{itemize}\} & 
				\begin{tabular}{P}
					TSP command: trigger.timer[n].start.stimulus \\
					n: timer number (1 to 4) \\
					This attribute shows the event that starts the trigger timer.
				\end{tabular} \\ \hline
		%---
		Timer{\textless n\textgreater}Over-Mon & bool\{\begin{itemize}[noitemsep]
					\small
					\item[] No overrun
					\item[] Overrun
				\end{itemize}\} & 
				\begin{tabular}{P}
					TSP command: trigger.timer[n].start.overrun \\
					n: timer number (1 to 4) \\
					This attribute indicates if an event was ignored because of the event detector state.
				\end{tabular} \\ \hline
		%---
		Timer{\textless n\textgreater}Clr-Cmd & bool\{\begin{itemize}[noitemsep]
					\small
					\item[] OFF
					\item[] ON
				\end{itemize}\} & 
				\begin{tabular}{P}
					TSP command: trigger.timer[n].clear() \\
					n: timer number (1 to 4) \\
					This function clears the timer event detector and overrun indicator for the corresponding trigger timer number.
				\end{tabular} \\ \hline
	\end{longtable}
	% TABLE: Blender
	\begin{longtable}{A B P}
		\caption{Blender} \\ \hline
		\bfseries Name & \bfseries Data Type & \bfseries Description \\ \hline
		%---
		Blend{\textless n\textgreater}Op-Sel & bool\{\begin{itemize}[noitemsep]
					\small
					\item[] OR
					\item[] AND
				\end{itemize}\} & 
				\begin{tabular}{P}
					TSP command: trigger.blender[n].orenable \\
					n: blender number (1 or 2) \\
					This attribute selects whether the blender performs OR operations or AND operations.
				\end{tabular} \\

		Blend{\textless n\textgreater}Op-Sts & bool\{\begin{itemize}[noitemsep]
					\small
					\item[] OR
					\item[] AND
				\end{itemize}\} & 
				\begin{tabular}{P}
					TSP command: trigger.blender[n].orenable \\
					n: blender number (1 or 2) \\
					This attribute shows the blender operation type.
				\end{tabular} \\ \hline
		%---
		Blend{\textless n\textgreater}Stim{\textless m\textgreater}-Sel & enum\{\begin{itemize}[noitemsep]
					\small
					\item[] EVENT\_NONE
					\item[] EVENT\_DISPLAY
					\item[] EVENT\_NOTIFY\textless i\textgreater
					\item[] ($1\leq i\leq 8$)
					\item[] EVENT\_COMMAND
					\item[] EVENT\_DIGIO\textless i\textgreater
					\item[] ($1\leq i\leq 6$)
					\item[] EVENT\_TSPLINK\textless i\textgreater
					\item[] ($1\leq i\leq 3$)
					\item[] EVENT\_LAN\textless i\textgreater
					\item[] ($1\leq i\leq 8$)
					\item[] EVENT\_BLENDER\textless i\textgreater 
					\item[] ($1\leq i\leq 2$)
					\item[] EVENT\_TIMER\textless i\textgreater
					\item[] ($1\leq i\leq 4$)
					\item[] EVENT\_ANALOGTRIGGER
					\item[] EVENT\_EXTERNAL
				\end{itemize}\} & 
				\begin{tabular}{P}
					TSP command: trigger.blender[n].stimulus[m] \\
					n: blender number (1 or 2) \\
					m: stimulus number (1 to 4) \\
					This attribute specifies the events that trigger the blender.
				\end{tabular} \\

		Blend{\textless n\textgreater}Stim{\textless m\textgreater}-Sts & enum\{\begin{itemize}[noitemsep]
					\small
					\item[] EVENT\_NONE
					\item[] EVENT\_DISPLAY
					\item[] EVENT\_NOTIFY\textless i\textgreater
					\item[] ($1\leq i\leq 8$)
					\item[] EVENT\_COMMAND
					\item[] EVENT\_DIGIO\textless i\textgreater
					\item[] ($1\leq i\leq 6$)
					\item[] EVENT\_TSPLINK\textless i\textgreater
					\item[] ($1\leq i\leq 3$)
					\item[] EVENT\_LAN\textless i\textgreater
					\item[] ($1\leq i\leq 8$)
					\item[] EVENT\_BLENDER\textless i\textgreater 
					\item[] ($1\leq i\leq 2$)
					\item[] EVENT\_TIMER\textless i\textgreater
					\item[] ($1\leq i\leq 4$)
					\item[] EVENT\_ANALOGTRIGGER
					\item[] EVENT\_EXTERNAL
				\end{itemize}\} & 
				\begin{tabular}{P}
					TSP command: trigger.blender[n].stimulus[m] \\
					n: blender number (1 or 2) \\
					m: stimulus number (1 to 4) \\
					This attribute shows the events that trigger the blender.
				\end{tabular} \\ \hline
		%---
		Blend{\textless n\textgreater}Over-Mon & bool\{\begin{itemize}[noitemsep]
					\small
					\item[] No overrun
					\item[] Overrun
				\end{itemize}\} & 
				\begin{tabular}{P}
					TSP command: trigger.blender[n].overrun \\
					n: blender number (1 or 2) \\
					This attribute indicates whether or not an event was ignored because of the event detector state.
				\end{tabular} \\ \hline
		%---
		Blend{\textless n \textgreater}Clr-Cmd & bool\{\begin{itemize}[noitemsep]
					\small
					\item[] OFF
					\item[] ON
				\end{itemize}\} & 
				\begin{tabular}{P}
					TSP command: trigger.blender[n].clear() \\
					n: blender number (1 or 2) \\
					This function clears the blender event detector and resets the overrun indicator of blender\textless n \textgreater.
				\end{tabular} \\ \hline
	\end{longtable}
	% TABLE: Autocalibration
	\begin{longtable}{A B P}
		\caption{Autocalibration} \\ \hline
		\bfseries Name & \bfseries Data Type & \bfseries Description \\ \hline
		%---
		ACalStart-Cmd & bool\{\begin{itemize}[noitemsep]
					\small
					\item[] OFF
					\item[] ON
				\end{itemize}\} & 
				\begin{tabular}{P}
					TSP command: acal.run() \\
					When set to 1 or \emph{ON}, this command immediately runs auto calibration and stores the constants.
				\end{tabular} \\ \hline
		%---
		ACalRev-Cmd & bool\{\begin{itemize}[noitemsep]
					\small
					\item[] OFF
					\item[] ON
				\end{itemize}\} & 
				\begin{tabular}{P}
					TSP command: acal.revert() \\
					When set to 1 or \emph{ON}, this command returns auto calibration constants to the previous constants.
				\end{tabular} \\ \hline
		%---
		ACalLast-Mon & \begin{tabular}{B}
					string 
				\end{tabular} & 
				\begin{tabular}{P}
					TSP command: acal.lastrun.time \\
					This variable returns, when processed, the date and time when auto calibration was last run. Set \emph{ACalLast-Mon.PROC} to any value in order to get the last calibration time once. Set \emph{ACalLast-Mon.SCAN} to an EPICS SCAN valide value in order to monitor the last calibration time periodically.
				\end{tabular} \\ \hline
		%---
		ACalCount-Mon & \begin{tabular}{B}
					long 
				\end{tabular} & 
				\begin{tabular}{P}
					TSP command: acal.count \\
					This attribute returns, when processed, the number of times automatic calibration has been run. Set \emph{ACalCount-Mon.PROC} to any value in order to get the automatic calibration count once. Set \emph{ACalCount-Mon.SCAN} to an EPICS SCAN valide value in order to monitor the automatic calibration count periodically.
				\end{tabular} \\ \hline
		%---
		ACalSchAct-Sel & enum\{\begin{itemize}[noitemsep]
					\small
					\item[] NONE
					\item[] Run
					\item[] Notify
				\end{itemize}\} & 
				\begin{tabular}{P}
					TSP command: acal.schedule() \\
					This attribute sets the action to be executed at the scheduled time.
				\end{tabular} \\

		ACalSchAct-Sts & enum\{\begin{itemize}[noitemsep]
					\small
					\item[] NONE
					\item[] Run
					\item[] Notify
				\end{itemize}\} & 
				\begin{tabular}{P}
					TSP command: acal.schedule() \\
					This attribute shows the action configured to be executed at the scheduled time.
				\end{tabular} \\ \hline
		%---
		ACalSchInt-Sel & bool\{\begin{itemize}[noitemsep]
					\small
					\item[] 8 hours
					\item[] 16 hours
					\item[] 1 day
					\item[] 7 days
					\item[] 14 days
					\item[] 30 days
					\item[] 90 days
				\end{itemize}\} & 
				\begin{tabular}{P}
					TSP command: acal.schedule() \\
					This attribute determines how often the auto calibration action should be executed.
				\end{tabular} \\

		ACalSchInt-Sts & bool\{\begin{itemize}[noitemsep]
					\small
					\item[] 8 hours
					\item[] 16 hours
					\item[] 1 day
					\item[] 7 days
					\item[] 14 days
					\item[] 30 days
					\item[] 90 days
				\end{itemize}\} & 
				\begin{tabular}{P}
					TSP command: acal.schedule() \\
					This attribute shows how often the auto calibration action is configured to be executed.
				\end{tabular} \\ \hline
		%---
		ACalSchHr-SP & \begin{tabular}{B}
					long \\
					Min=0 \\
					Max=23
				\end{tabular} & 
				\begin{tabular}{P}
					TSP command: acal.schedule() \\
					Specify, in 24-hour time format, when the auto calibration action should occur.
				\end{tabular} \\

		ACalSchHr-RB & \begin{tabular}{B}
					long
				\end{tabular} & 
				\begin{tabular}{P}
					TSP command: acal.schedule() \\
					Shows, in 24-hour format, the configured time for the auto calibration action.
				\end{tabular} \\ \hline
		%---
		ACalNext-Mon & \begin{tabular}{B}
					string
				\end{tabular} & 
				\begin{tabular}{P}
					TSP command: acal.nextrun.time \\
					This attribute returns, when processed, the date and time when the next auto calibration is scheduled to be run. Set \emph{ACalNext-Mon.PROC} to any value in order to get the next auto calibration date and time once. Set \emph{ACalNext-Mon.SCAN} to an EPICS SCAN valide value in order to monitor the next auto calibration date and time periodically.
				\end{tabular} \\ \hline
		%---
		ACalDiff-Mon & \begin{tabular}{B}
					float
				\end{tabular} & 
				\begin{tabular}{P}
					TSP command: acal.lastrun.tempdiff \\
					When processed, this attribute returns the difference between the internal temperature and the temperature when auto calibration was last run. Set \emph{ACalDiff-Mon.PROC} to any value in order to read the temperature difference once. Set \emph{ACalDiff-Mon.SCAN} to an EPICS SCAN valide value in order to monitor the difference between the current internal temperature and the temperature when auto calibration was last run, periodically.
				\end{tabular} \\ \hline
		%---
		ACalLim-SP & \begin{tabular}{B}
					float
				\end{tabular} & 
				\begin{tabular}{P}
					TSP command: No command \\
					This sets the maximum accepted instrument internal temperature variation. When the variation exceeds the specified value, the \emph{ACalWarn-Mon} PV is set to 1.
				\end{tabular} \\ \hline
		%---
		ACalWarn-Mon & \begin{tabular}{B}
					float
				\end{tabular} & 
				\begin{tabular}{P}
					TSP command: No command \\
					This variable indicates when the instrument internal temperature variation has exceeded the value specified by \emph{ACalLim-SP}. When the limit is exceeded, the variable is set to 1 until a \emph{warning reset} is performed (\emph{ACalRst-Cmd}).
				\end{tabular} \\ \hline
		%---
		ACalRst-Cmd & bool\{\begin{itemize}[noitemsep]
					\small
					\item[] OFF
					\item[] ON
				\end{itemize}\} & 
				\begin{tabular}{P}
					TSP command: No command \\
					When set to 1 or \emph{ON}, this command resets the autocalibration warning, i.e., the \emph{ACalWarn-Mon} PV is set to 0.
				\end{tabular} \\ \hline
	\end{longtable}
	% TABLE: Trigger Model
	\begin{longtable}{A B P}
		\caption{Trigger Model} \\ \hline
		\bfseries Name & \bfseries Data Type & \bfseries Description \\ \hline
		%---
		TMStart-Cmd & bool\{\begin{itemize}[noitemsep]
					\small
					\item[] OFF
					\item[] ON
				\end{itemize}\} & 
				\begin{tabular}{P}
					TSP command: trigger.model.initiate() \\
					When set to 1 or \emph{ON}, this command starts the trigger model.
				\end{tabular} \\ \hline
		%---
		TMAbort-Cmd & bool\{\begin{itemize}[noitemsep]
					\small
					\item[] OFF
					\item[] ON
				\end{itemize}\} & 
				\begin{tabular}{P}
					TSP command: trigger.model.abort() \\
					When set to 1 or \emph{ON}, this command stops all trigger model commands on the instrument.
				\end{tabular} \\ \hline
		%---
		TMClear-Cmd & bool\{\begin{itemize}[noitemsep]
					\small
					\item[] OFF
					\item[] ON
				\end{itemize}\} & 
				\begin{tabular}{P}
					TSP command: trigger.model.load("Empty") \\
					When set to 1 or \emph{ON}, this command clears the trigger model.
				\end{tabular} \\ \hline
		%---
		TM-Mon & enum\{\begin{itemize}[noitemsep]
					\small
					\item[] Idle
					\item[] Running
					\item[] Waiting
					\item[] Empty
					\item[] Building
					\item[] Failed
					\item[] Aborting
					\item[] Aborted
				\end{itemize}\} & 
				\begin{tabular}{P}
					TSP command: trigger.model.state() \\
					When processed, this variable reads the present state of the trigger model. Set \emph{TM-Mon.PROC} to any value in order to read the trigger model state once. Set \emph{TM-Mon.SCAN} to an EPICS SCAN valide value in order to monitor the trigger model state periodically.
				\end{tabular} \\ \hline
		%---
		TMBlockList-Mon & \begin{tabular}{B}
					char[2500]
				\end{tabular} & 
				\begin{tabular}{P}
					TSP command: trigger.model.getblocklist() \\
					When processed, this variables reads the settings for all trigger model blocks. Set \emph{TMBlockList-Mon.PROC} to any value in order to read the trigger model settings once. Set \emph{TMBlockList-Mon.SCAN} to an EPICS SCAN valide value in order to monitor the trigger model settings periodically.
				\end{tabular} \\ \hline
	\end{longtable}
	% TABLE: General
	\begin{longtable}{A B P}
		\caption{General} \\ \hline
		\bfseries Name & \bfseries Data Type & \bfseries Description \\ \hline
		%---
		Reset-Cmd & bool\{\begin{itemize}[noitemsep]
					\small
					\item[] 
					\item[] 
				\end{itemize}\} & 
				\begin{tabular}{P}
					TSP command: reset() \\
					When set to 1 or \emph{ON}, this command resets parameters to their default settings and clears the buffers.
				\end{tabular} \\ \hline
		%---
		Access-Sel & bool\{\begin{itemize}[noitemsep]
					\small
					\item[] FULL
					\item[] EXCLUSIVE
					\item[] PROTECTED
					\item[] LOCKOUT
				\end{itemize}\} & 
				\begin{tabular}{P}
					TSP command: localnode.access \\
					This attribute defines the type of access users have to the instrument through different interfaces.
				\end{tabular} \\

		Access-Sts & bool\{\begin{itemize}[noitemsep]
					\small
					\item[] FULL
					\item[] EXCLUSIVE
					\item[] PROTECTED
					\item[] LOCKOUT
				\end{itemize}\} & 
				\begin{tabular}{P}
					TSP command: localnode.access \\
					This attribute shows the type of access users have to the instrument through different interfaces.
				\end{tabular} \\ \hline
		%---
		Login-SP & \begin{tabular}{B}
					string
				\end{tabular} & 
				\begin{tabular}{P}
					TSP command: login \\
					This variable sends a login command to the instrument using the password entered.
				\end{tabular} \\ \hline
		%---
		Logout-Cmd & bool\{\begin{itemize}[noitemsep]
					\small
					\item[] OFF
					\item[] ON
				\end{itemize}\} & 
				\begin{tabular}{P}
					TSP command: logout \\
					When set to 1 or \emph{ON}, this command sends a logout command to the instrument.
				\end{tabular} \\ \hline
		%---
		PassNew-SP & \begin{tabular}{B}
					string
				\end{tabular} & 
				\begin{tabular}{P}
					TSP command: localnode.password \\
					This attribute sets the instrument password.
				\end{tabular} \\ \hline
		%---
		Time-SP & \begin{tabular}{B}
					string \\
					Format: \textless year\textgreater, \textless month\textgreater, \textless day\textgreater, \textless hour\textgreater, \textless minute\textgreater, \textless second\textgreater
				\end{tabular} & 
				\begin{tabular}{P}
					TSP command: localnode.settime() \\
					This attribute sets the date and time of the instrument.
				\end{tabular} \\ \hline
		%---
		Time-Mon & \begin{tabular}{B}
					string \\
					Format: \textless day of the week\textgreater, \textless month\textgreater, \textless day\textgreater, \textless hour\textgreater, \textless minute\textgreater, \textless second\textgreater, \textless year\textgreater
				\end{tabular} & 
				\begin{tabular}{P}
					TSP command: print(os.date('\%c', gettime())) \\
					When processed, this variable reads the date and time of the instrument. Set \emph{Time-Mon.PROC} to any value in order to read the instrument date and time once. Set \emph{Time-Mon.SCAN} to an EPICS SCAN valide value in order to monitor the instrument date and time periodically.
				\end{tabular} \\ \hline
		%---
		EvLogCount-Mon & \begin{tabular}{B}
					long
				\end{tabular} & 
				\begin{tabular}{P}
					TSP command: print(eventlog.getcount()) \\
					When processed, this variable reads the number of unread events in the event log. Set \emph{EvLogCount-Mon.PROC} to any value in order to read the number of unread events once. Set \emph{EvLogCount-Mon.SCAN} to an EPICS SCAN valide value in order to monitor the number of unread events periodically.
				\end{tabular} \\ \hline
		%---
		EvLogNext-Mon & \begin{tabular}{B}
					char[250]
				\end{tabular} & 
				\begin{tabular}{P}
					TSP command: eventlog.next() \\
					When processed, this variable reads the oldest unread event message from the event log. Set \emph{EvLogNext-Mon.PROC} to any value in order to read the oldest unread event message once. Set \emph{EvLogNext-Mon.SCAN} to an EPICS SCAN valide value in order to fetch events periodically.
				\end{tabular} \\ \hline
		%---
		ClearEvLog-Cmd & bool\{\begin{itemize}[noitemsep]
					\small
					\item[] OFF
					\item[] ON
				\end{itemize}\} & 
				\begin{tabular}{P}
					TSP command: eventlog.clear() \\
					When set to 1 or \emph{ON}, this command clears the event log.
				\end{tabular} \\ \hline
		%---
		LineFR-Mon & \begin{tabular}{B}
					long
				\end{tabular} & 
				\begin{tabular}{P}
					TSP command: print(localnode.linefreq) \\
					When processed, this variable reads the power line frequency setting that is used for NPLC calculations. Set \emph{LineFR-Mon.PROC} to any value in order to read the power line frequency setting once.
				\end{tabular} \\ \hline
		%---
		Temp-Mon & \begin{tabular}{B}
					float
				\end{tabular} & 
				\begin{tabular}{P}
					TSP command: print(localnode.internaltemp) \\
					When processed, this variable reads the internal temperature of the instrument. Set \emph{Temp-Mon.PROC} to any value in order to read the instrument internal temperature once. Set \emph{Temp-Mon.SCAN} to an EPICS SCAN valide value in order to monitor the instrument internal temperature periodically.
				\end{tabular} \\ \hline
		%---
		TimeSec-Mon & \begin{tabular}{B}
					long
				\end{tabular} & 
				\begin{tabular}{P}
					TSP command: print(localnode.gettime()) \\
					This variable monitors the instrument time in order to periodically check the connection status. The connection status is display by the variable \emph{Network-Mon}. In order to disable connection monitoring, set \emph{TimeSec-Mon.SCAN} to \emph{Passive}.
				\end{tabular} \\ \hline
		%---
		Network-Mon & bool\{\begin{itemize}[noitemsep]
					\small
					\item[] OFF
					\item[] ON
				\end{itemize}\} & 
				\begin{tabular}{P}
					TSP command: No command \\
					This variable displays the status of the connection between the instrument and the PC running the IOC. In order to disable connection monitoring, set \emph{TimeSec-Mon.SCAN} to \emph{Passive}.
				\end{tabular} \\ \hline
		%---
		Upload-Cmd & bool\{\begin{itemize}[noitemsep]
					\small
					\item[] OFF
					\item[] ON
				\end{itemize}\} & 
				\begin{tabular}{P}
					TSP command: No command \\
					When set to 1 or \emph{ON}, this command updates all readback and status PVs (-RB and -Sts) of the IOC. An update happens automatically when the IOC is started up, or when the network connection is lost and reconnects. The later requires \emph{TimeSec-Mon.SCAN} to be different from \emph{Passive}.
				\end{tabular} \\ \hline
		%---
		Custom-SP & \begin{tabular}{B}
					char[250]
				\end{tabular} & 
				\begin{tabular}{P}
					TSP command: Any command passed as a string \\
					This variable is an array that can send any string as a command to the instrument, provided that the string length does not exceed the array length.
				\end{tabular} \\ \hline
	\end{longtable}

\end{document}
\grid
